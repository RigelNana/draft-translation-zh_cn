\rSec0[concepts]{概念库}

\rSec1[concepts.general]{概述}

\pnum
本条款描述了 \Cpp{} 程序可以用来执行
模板参数的编译时验证并执行函数分派的库组件,
基于类型的属性。这些概念的目的是建立
程序中等式推理的基础。

\pnum
以下子条款描述了语言相关的概念、比较
概念、对象概念和可调用概念,如
\tref{concepts.summary} 中所总结。

\begin{libsumtab}{基本概念库摘要}{concepts.summary}
\ref{concepts.equality} & 等式保持     &                    \\ \hline
\ref{concepts.lang}     & 语言相关概念 & \tcode{<concepts>} \\
\ref{concepts.compare}  & 比较概念       &                    \\
\ref{concepts.object}   & 对象概念           &                    \\
\ref{concepts.callable} & 可调用概念         &                    \\
\end{libsumtab}

\rSec1[concepts.equality]{等式保持}

\pnum
如果给定相等的输入,表达式会产生相等的结果,则表达式是 \defnx{等式保持}{expression!equality-preserving} 的。
表达式的输入是表达式操作数的集合。
表达式的输出是表达式的结果和表达式修改的所有操作数。
对于本子条款的目的,
表达式的操作数是包含以下内容的最大子表达式:
\begin{itemize}
\item
\grammarterm{id-expression}\iref{expr.prim.id},和
\item
库函数模板的调用
\tcode{std::move}、
\tcode{std::forward} 和
\tcode{std::declval}\iref{forward,declval}。
\end{itemize}
\begin{example}
表达式 \tcode{a = std::move(b)} 的操作数是
\tcode{a} 和 \tcode{std::move(b)}。
\end{example}

\pnum
并非所有输入值对于给定表达式都需要有效。
\begin{example}
对于整数 \tcode{a} 和 \tcode{b},
当 \tcode{b} 为 \tcode{0} 时,表达式 \tcode{a / b} 没有明确定义。
这并不排除表达式 \tcode{a / b} 是等式保持的。
\end{example}
表达式的 \defnx{域}{expression!domain} 是
需要将表达式定义为良构的输入值的集合。

\pnum
需要保持等式的表达式进一步
需要是稳定的:在没有任何明确干预的情况下,使用相同的输入对象对该表达式进行两次求值
需要修改这些输入对象才能得到相等的输出。
\begin{note}
此要求允许通用代码基于
通过等式保持表达式观察到的先前值来推断对象的当前值。它有效地禁止自发更改
对象、来自另一个执行线程的更改、作为非修改表达式的副作用对对象的更改以及
如果这些更改可以通过需要有效的等式保持表达式对库函数可见,则作为修改不同对象的副作用对对象的更改
为该对象。
\end{note}

\pnum
库条款中的 \grammarterm{requires-expression} 中声明的表达式是
需要保持等式的,除非那些带有注释
“不需要保持等式”的表达式。如此注释的表达式
可能是等式保持的,但不是必需的。

\pnum
可能会改变其一个或多个输入的值,
其方式对等式保持表达式是可观察的,则称该表达式修改了这些输入。
库条款使用符号约定来指定哪些表达式在
\grammarterm{requires-expression} 中声明修改哪些输入:除非
另有说明,否则作为非常量左值或
右值的表达式操作数可能会被修改。要求不修改作为常量左值或右值的操作数。
对于本子条款的目的,
每个操作数的 cv 限定和值类别
由假设
每个模板类型参数
表示一个 cv 非限定的完整非数组对象类型。

\pnum
当 \grammarterm{requires-expression} 声明一个表达式是
对于某些常量左值操作数是非修改的,该表达式的其他变体接受非常量左值或(可能为常量)右值
对于给定的操作数也是必需的,除非明确要求此类表达式变体具有不同的语义。这些
\term{隐式表达式变体} 需要满足
声明的表达式的语义要求。实现到何种程度
验证变体的语法是未指定的。

\pnum
\begin{example}
\begin{codeblock}
template<class T> concept C = requires(T a, T b, const T c, const T d) {
  c == d;           // \#1
  a = std::move(b); // \#2
  a = c;            // \#3
};
\end{codeblock}

对于上面的例子:
\begin{itemize}
\item
表达式 \#1 不修改其任何一个操作数,\#2 修改其两个操作数,
而 \#3 仅修改其第一个操作数 \tcode{a}。

\item
表达式 \#1 隐式地要求满足
\tcode{c == d}(包括非修改)要求的附加表达式变体,就好像
表达式
\begin{codeblock}
                                            c  ==           b;
          c  == std::move(d);               c  == std::move(b);
std::move(c) ==           d;      std::move(c) ==           b;
std::move(c) == std::move(d);     std::move(c) == std::move(b);

          a  ==           d;                a  ==           b;
          a  == std::move(d);               a  == std::move(b);
std::move(a) ==           d;      std::move(a) ==           b;
std::move(a) == std::move(d);     std::move(a) == std::move(b);
\end{codeblock}
也已被声明。

\item
表达式 \#3 隐式地要求满足
\tcode{a = c}(包括不修改第二个操作数)要求的附加表达式变体,就好像表达式 \tcode{a = b} 和 \tcode{a = std::move(c)} 已
被声明。表达式 \#3 不隐式地要求使用非常量右值第二个操作数的表达式
变体,因为表达式 \#2 已经明确指定了这样的表达式。
\end{itemize}
\end{example}

\pnum
\begin{example}
以下类型 \tcode{T} 满足
上面概念 \tcode{C} 的显式声明的语法要求,但不满足附加的隐式
要求:

\begin{codeblock}
struct T {
  bool operator==(const T&) const { return true; }
  bool operator==(T&) = delete;
};
\end{codeblock}

\tcode{T} 未能满足 \tcode{C} 的隐式要求,
因此 \tcode{T} 满足但不建模 \tcode{C}。
由于不需要实现来验证隐式要求的语法,
因此未指定实现是否将需要 \tcode{C<T>} 的程序诊断为病态的。
\end{example}

\rSec1[concepts.syn]{头文件 \tcode{<concepts>} 概要}

\indexheader{concepts}%
\begin{codeblock}
// all freestanding
namespace std {
  // \ref{concepts.lang},语言相关概念
  // \ref{concept.same},概念 \libconcept{same_as}
  template<class T, class U>
    concept same_as = @\seebelow@;

  // \ref{concept.derived},概念 \libconcept{derived_from}
  template<class Derived, class Base>
    concept derived_from = @\seebelow@;

  // \ref{concept.convertible},概念 \libconcept{convertible_to}
  template<class From, class To>
    concept convertible_to = @\seebelow@;

  // \ref{concept.commonref},概念 \libconcept{common_reference_with}
  template<class T, class U>
    concept common_reference_with = @\seebelow@;

  // \ref{concept.common},概念 \libconcept{common_with}
  template<class T, class U>
    concept common_with = @\seebelow@;

  // \ref{concepts.arithmetic},算术概念
  template<class T>
    concept integral = @\seebelow@;
  template<class T>
    concept signed_integral = @\seebelow@;
  template<class T>
    concept unsigned_integral = @\seebelow@;
  template<class T>
    concept floating_point = @\seebelow@;

  // \ref{concept.assignable},概念 \libconcept{assignable_from}
  template<class LHS, class RHS>
    concept assignable_from = @\seebelow@;

  // \ref{concept.swappable},概念 \libconcept{swappable}
  namespace ranges {
    inline namespace @\unspec@ {
      inline constexpr @\unspec@ swap = @\unspec@;
    }
  }
  template<class T>
    concept swappable = @\seebelow@;
  template<class T, class U>
    concept swappable_with = @\seebelow@;

  // \ref{concept.destructible},概念 \libconcept{destructible}
  template<class T>
    concept destructible = @\seebelow@;

  // \ref{concept.constructible},概念 \libconcept{constructible_from}
  template<class T, class... Args>
    concept constructible_from = @\seebelow@;

  // \ref{concept.default.init},概念 \libconcept{default_initializable}
  template<class T>
    concept default_initializable = @\seebelow@;

  // \ref{concept.moveconstructible},概念 \libconcept{move_constructible}
  template<class T>
    concept move_constructible = @\seebelow@;

  // \ref{concept.copyconstructible},概念 \libconcept{copy_constructible}
  template<class T>
    concept copy_constructible = @\seebelow@;

  // \ref{concepts.compare},比较概念
  // \ref{concept.equalitycomparable},概念 \libconcept{equality_comparable}
  template<class T>
    concept equality_comparable = @\seebelow@;
  template<class T, class U>
    concept equality_comparable_with = @\seebelow@;

  // \ref{concept.totallyordered},概念 \libconcept{totally_ordered}
  template<class T>
    concept totally_ordered = @\seebelow@;
  template<class T, class U>
    concept totally_ordered_with = @\seebelow@;

  // \ref{concepts.object},对象概念
  template<class T>
    concept movable = @\seebelow@;
  template<class T>
    concept copyable = @\seebelow@;
  template<class T>
    concept semiregular = @\seebelow@;
  template<class T>
    concept regular = @\seebelow@;

  // \ref{concepts.callable},可调用概念
  // \ref{concept.invocable},概念 \libconcept{invocable}
  template<class F, class... Args>
    concept invocable = @\seebelow@;

  // \ref{concept.regularinvocable},概念 \libconcept{regular_invocable}
  template<class F, class... Args>
    concept regular_invocable = @\seebelow@;

  // \ref{concept.predicate},概念 \libconcept{predicate}
  template<class F, class... Args>
    concept predicate = @\seebelow@;

  // \ref{concept.relation},概念 \libconcept{relation}
  template<class R, class T, class U>
    concept relation = @\seebelow@;

  // \ref{concept.equiv},概念 \libconcept{equivalence_relation}
  template<class R, class T, class U>
    concept equivalence_relation = @\seebelow@;

  // \ref{concept.strictweakorder},概念 \libconcept{strict_weak_order}
  template<class R, class T, class U>
    concept strict_weak_order = @\seebelow@;
}
\end{codeblock}

\rSec1[concepts.lang]{语言相关概念}

\rSec2[concepts.lang.general]{概述}

\pnum
子条款 \ref{concepts.lang} 包含对应于语言特性的概念定义。
这些概念表达了类型、类型分类和基本类型属性之间的关系。

\rSec2[concept.same]{概念 \cname{same_as}}

\begin{itemdecl}
template<class T, class U>
  concept @\defexposconcept{same-as-impl}@ = is_same_v<T, U>;       // \expos

template<class T, class U>
  concept @\deflibconcept{same_as}@ = @\exposconcept{same-as-impl}@<T, U> && @\exposconcept{same-as-impl}@<U, T>;
\end{itemdecl}

\begin{itemdescr}
\pnum
\begin{note}
\tcode{\libconcept{same_as}<T, U>} 包含 \tcode{\libconcept{same_as}<U, T>} 并且
反之亦然。
\end{note}
\end{itemdescr}

\rSec2[concept.derived]{概念 \cname{derived_from}}

\begin{itemdecl}
template<class Derived, class Base>
  concept @\deflibconcept{derived_from}@ =
    is_base_of_v<Base, Derived> &&
    is_convertible_v<const volatile Derived*, const volatile Base*>;
\end{itemdecl}

\begin{itemdescr}
\pnum
\begin{note}
\tcode{\libconcept{derived_from}<Derived, Base>} 当且仅当
\tcode{Derived} 公开且明确地派生自 \tcode{Base} 时才满足,或者
\tcode{Derived} 和 \tcode{Base} 是相同的类类型,忽略 cv 限定符。
\end{note}
\end{itemdescr}

\rSec2[concept.convertible]{概念 \cname{convertible_to}}

\pnum
给定类型 \tcode{From} 和 \tcode{To} 以及
类型和值类别与 \tcode{declval<From>()} 相同的表达式 \tcode{E},
\tcode{\libconcept{convertible_to}<From, To>} 要求 \tcode{E}
可以隐式和显式地转换为类型 \tcode{To}。
隐式和显式转换需要产生相等的结果。

\begin{itemdecl}
template<class From, class To>
  concept @\deflibconcept{convertible_to}@ =
    is_convertible_v<From, To> &&
    requires {
      static_cast<To>(declval<From>());
    };
\end{itemdecl}

\begin{itemdescr}
\pnum
令 \tcode{FromR} 为 \tcode{add_rvalue_reference_t<From>} 并且
\tcode{test} 为虚构的函数:
\begin{codeblock}
To test(FromR (&f)()) {
  return f();
}
\end{codeblock}
令 \tcode{f} 为一个没有参数且返回类型为 \tcode{FromR} 的函数
使得 \tcode{f()} 是等式保持的。
类型 \tcode{From} 和 \tcode{To} 建模 \tcode{\libconcept{convertible_to}<From, To>}
仅当:

\begin{itemize}
\item
\tcode{To} 不是对象或对象引用类型,或者
\tcode{static_cast<To>(f())} 等于 \tcode{test(f)}。

\item
\tcode{FromR} 不是对象引用类型,或者

\begin{itemize}
\item
如果 \tcode{FromR} 是对非 const 限定类型的右值引用,则
在上述任一表达式之后,\tcode{f()} 引用的对象的状态
是有效的但未指定的\iref{lib.types.movedfrom}。

\item
否则,\tcode{f()} 引用的对象不会被上述任一表达式修改。
\end{itemize}
\end{itemize}
\end{itemdescr}


\rSec2[concept.commonref]{概念 \cname{common_reference_with}}

\pnum
对于两种类型 \tcode{T} 和 \tcode{U},如果 \tcode{common_reference_t<T, U>}
是良构的并且表示一个类型 \tcode{C},使得
\tcode{\libconcept{convertible_to}<T, C>}
和
\tcode{\libconcept{convertible_to}<U, C>}
均被建模,则 \tcode{T} 和 \tcode{U} 共享一个
\term{共同引用类型},\tcode{C}。
\begin{note}
\tcode{C} 可以与 \tcode{T} 或 \tcode{U} 相同,或者可以是
不同的类型。\tcode{C} 可以是引用类型。
\end{note}

\begin{itemdecl}
template<class T, class U>
  concept @\deflibconcept{common_reference_with}@ =
    @\libconcept{same_as}@<common_reference_t<T, U>, common_reference_t<U, T>> &&
    @\libconcept{convertible_to}@<T, common_reference_t<T, U>> &&
    @\libconcept{convertible_to}@<U, common_reference_t<T, U>>;
\end{itemdecl}

\begin{itemdescr}
\pnum
令 \tcode{C} 为 \tcode{common_reference_t<T, U>}。
令 \tcode{t1} 和 \tcode{t2} 为等式保持
表达式\iref{concepts.equality},使得
\tcode{decltype((t1))} 和 \tcode{decltype((t2))} 均为 \tcode{T},并且
令 \tcode{u1} 和 \tcode{u2} 为等式保持表达式,使得
\tcode{decltype((u1))} 和 \tcode{decltype((u2))} 均为 \tcode{U}。
\tcode{T} 和 \tcode{U} 建模 \tcode{\libconcept{common_reference_with}<T, U>}
仅当
\begin{itemize}
\item \tcode{C(t1)} 等于 \tcode{C(t2)} 当且仅当
  \tcode{t1} 等于 \tcode{t2},并且
\item \tcode{C(u1)} 等于 \tcode{C(u2)} 当且仅当
  \tcode{u1} 等于 \tcode{u2}。
\end{itemize}

\pnum
\begin{note}
用户可以通过特化
\tcode{basic_common_reference} 类模板\iref{meta.trans.other} 来定制 \libconcept{common_reference_with} 的行为。
\end{note}
\end{itemdescr}

\rSec2[concept.common]{概念 \cname{common_with}}

\pnum
如果 \tcode{T} 和 \tcode{U} 都可以显式转换为某个第三类型,
\tcode{C},则 \tcode{T} 和 \tcode{U} 共享一个 \term{共同类型},
\tcode{C}。
\begin{note}
\tcode{C} 可以与 \tcode{T} 或 \tcode{U} 相同,或者可以是
不同的类型。\tcode{C} 不一定是唯一的。
\end{note}

\begin{itemdecl}
template<class T, class U>
  concept @\deflibconcept{common_with}@ =
    @\libconcept{same_as}@<common_type_t<T, U>, common_type_t<U, T>> &&
    requires {
      static_cast<common_type_t<T, U>>(declval<T>());
      static_cast<common_type_t<T, U>>(declval<U>());
    } &&
    @\libconcept{common_reference_with}@<
      add_lvalue_reference_t<const T>,
      add_lvalue_reference_t<const U>> &&
    @\libconcept{common_reference_with}@<
      add_lvalue_reference_t<common_type_t<T, U>>,
      common_reference_t<
        add_lvalue_reference_t<const T>,
        add_lvalue_reference_t<const U>>>;
\end{itemdecl}

\begin{itemdescr}
\pnum
令 \tcode{C} 为 \tcode{common_type_t<T, U>}。
令 \tcode{t1} 和 \tcode{t2} 为
等式保持表达式\iref{concepts.equality},使得
\tcode{decltype((t1))} 和 \tcode{decltype((t2))} 均为 \tcode{T},并且
令 \tcode{u1} 和 \tcode{u2} 为等式保持表达式,使得
\tcode{decltype((u1))} 和 \tcode{decltype((u2))} 均为 \tcode{U}。
\tcode{T} 和 \tcode{U} 建模 \tcode{\libconcept{common_with}<T, U>}
仅当
\begin{itemize}
\item \tcode{C(t1)} 等于 \tcode{C(t2)} 当且仅当
  \tcode{t1} 等于 \tcode{t2},并且
\item \tcode{C(u1)} 等于 \tcode{C(u2)} 当且仅当
  \tcode{u1} 等于 \tcode{u2}。
\end{itemize}

\pnum
\begin{note}
用户可以通过特化
\tcode{common_type} 类模板\iref{meta.trans.other} 来定制 \libconcept{common_with} 的行为。
\end{note}

\end{itemdescr}

\rSec2[concepts.arithmetic]{算术概念}

\begin{itemdecl}
template<class T>
  concept @\deflibconcept{integral}@ = is_integral_v<T>;
template<class T>
  concept @\deflibconcept{signed_integral}@ = @\libconcept{integral}@<T> && is_signed_v<T>;
template<class T>
  concept @\deflibconcept{unsigned_integral}@ = @\libconcept{integral}@<T> && !@\libconcept{signed_integral}@<T>;
template<class T>
  concept @\deflibconcept{floating_point}@ = is_floating_point_v<T>;
\end{itemdecl}

\begin{itemdescr}
\pnum
\begin{note}
\libconcept{signed_integral} 甚至可以由
不是有符号整数类型\iref{basic.fundamental} 的类型建模;例如,\tcode{char}。
\end{note}

\pnum
\begin{note}
\libconcept{unsigned_integral} 甚至可以由
不是无符号整数类型\iref{basic.fundamental} 的类型建模;例如,\tcode{bool}。
\end{note}
\end{itemdescr}

\rSec2[concept.assignable]{概念 \cname{assignable_from}}

\begin{itemdecl}
template<class LHS, class RHS>
  concept @\deflibconcept{assignable_from}@ =
    is_lvalue_reference_v<LHS> &&
    @\libconcept{common_reference_with}@<const remove_reference_t<LHS>&, const remove_reference_t<RHS>&> &&
    requires(LHS lhs, RHS&& rhs) {
      { lhs = std::forward<RHS>(rhs) } -> @\libconcept{same_as}@<LHS>;
    };
\end{itemdecl}

\begin{itemdescr}
\pnum
令:
\begin{itemize}
\item \tcode{lhs} 为一个左值,它引用一个对象 \tcode{lcopy},使得
  \tcode{decltype((lhs))} 为 \tcode{LHS},
\item \tcode{rhs} 为一个表达式,使得 \tcode{decltype((rhs))} 为
  \tcode{RHS},并且
\item \tcode{rcopy} 为一个与 \tcode{rhs} 相等的不同对象。
\end{itemize}
\tcode{LHS} 和 \tcode{RHS} 建模
\tcode{\libconcept{assignable_from}<LHS, RHS>} 仅当

\begin{itemize}
\item \tcode{addressof(lhs = rhs) == addressof(lcopy)}。

\item 在评估 \tcode{lhs = rhs} 之后:

\begin{itemize}
\item \tcode{lhs} 等于 \tcode{rcopy},除非 \tcode{rhs} 是一个引用 \tcode{lcopy} 的非常量
xvalue。

\item 如果 \tcode{rhs} 是一个非常量 xvalue,则其引用的对象的状态
是有效的但未指定的\iref{lib.types.movedfrom}。

\item 否则,如果 \tcode{rhs} 是一个 glvalue,则其引用的对象
  不被修改。
\end{itemize}
\end{itemize}

\pnum
\begin{note}
赋值不需要是全函数\iref{structure.requirements};
特别是,如果对对象 \tcode{x} 的赋值可能导致
修改某个其他对象 \tcode{y},则 \tcode{x = y} 很可能不在 \tcode{=} 的域中。
\end{note}
\end{itemdescr}

\rSec2[concept.swappable]{概念 \cname{swappable}}

\pnum
令 \tcode{t1} 和 \tcode{t2} 为表示
类型为 \tcode{T} 的不同相等对象的等式保持表达式,令 \tcode{u1} 和 \tcode{u2}
类似地表示类型为 \tcode{U} 的不同相等对象。
\begin{note}
\tcode{t1} 和 \tcode{u1} 可以表示不同的对象,或相同的对象。
\end{note}
当且仅当该操作
既不修改 \tcode{t2} 也不修改 \tcode{u2} 并且:
\begin{itemize}
\item 如果 \tcode{T} 和 \tcode{U} 是相同的类型,则操作的结果是
  \tcode{t1} 等于 \tcode{u2} 并且 \tcode{u1} 等于 \tcode{t2}。

\item 如果 \tcode{T} 和 \tcode{U} 是不同的类型并且
  \tcode{\libconcept{common_reference_with}<decltype((t1)), decltype((u1))>}
  被建模,
  操作的结果是
  \tcode{C(t1)} 等于 \tcode{C(u2)}
  并且
  \tcode{C(u1)} 等于 \tcode{C(t2)}
  其中 \tcode{C} 是 \tcode{common_reference_t<decltype((t1)), decltype((u1))>}。
\end{itemize}

\pnum
\indexlibraryglobal{swap}%
名称 \tcode{ranges::swap} 表示一个定制点
对象\iref{customization.point.object}。表达式
\tcode{ranges::swap(E1, E2)} 对于子表达式 \tcode{E1}
和 \tcode{E2} 与表达式
\tcode{S} 在表达式上等价,\tcode{S} 的确定方式如下:

\begin{itemize}
\item
  \tcode{S} 是 \tcode{(void)swap(E1, E2)}
\begin{footnote}
名称 \tcode{swap} 是
  此处不限定使用的。
\end{footnote}
如果 \tcode{E1} 或 \tcode{E2}
  具有类或枚举类型\iref{basic.compound} 并且该表达式有效,则在包含声明的上下文中执行重载决议
\begin{codeblock}
template<class T>
  void swap(T&, T&) = delete;
\end{codeblock}
  并且不包括 \tcode{ranges::swap} 的声明。
  如果重载决议选择的函数没有
  交换 \tcode{E1} 和 \tcode{E2} 所表示的值,
  则程序是病态的,不需要诊断。
  \begin{note}
  这排除了调用非约束的程序定义的 \tcode{swap} 重载。当删除的重载
  是可行的时,程序定义的重载需要更加特化\iref{temp.func.order} 才能被选择。
  \end{note}

\item
  否则,如果 \tcode{E1} 和 \tcode{E2}
  是具有相等范围的数组类型\iref{basic.compound} 的左值
  并且 \tcode{ranges::swap(*E1, *E2)}
  是一个有效表达式,
  则 \tcode{S} 是 \tcode{(void)ranges::swap_ranges(E1, E2)},
  除了
  \tcode{noexcept(S)} 等于
  \tcode{noexcept(\brk{}ranges::swap(*E1, *E2))}。

\item
  否则,如果 \tcode{E1} 和 \tcode{E2} 是
  相同类型 \tcode{T} 的左值,该类型建模 \tcode{\libconcept{move_constructible}<T>} 和
  \tcode{\libconcept{assignable_from}<T\&, T>},
  则 \tcode{S} 是一个交换表示值的表达式。
  \tcode{S} 是一个常量表达式,如果
  \begin{itemize}
  \item \tcode{T} 是一个字面量类型\iref{term.literal.type},
  \item \tcode{E1 = std::move(E2)} 和 \tcode{E2 = std::move(E1)} 均为
    常量子表达式\iref{defns.const.subexpr},并且
  \item 声明中初始值设定项的完整表达式
\begin{codeblock}
T t1(std::move(E1));
T t2(std::move(E2));
\end{codeblock}
是常量子表达式。
  \end{itemize}
  \tcode{noexcept(S)} 等于
  \tcode{is_nothrow_move_constructible_v<T> \&\& is_nothrow_move_assignable\-_v<T>}。

\item
  否则,\tcode{ranges::swap(E1, E2)} 是病态的。
  \begin{note}
  当 \tcode{ranges::swap(E1, E2)} 出现在模板实例化的直接上下文中时,这种情况可能导致替换失败。
  \end{note}
\end{itemize}

\pnum
\begin{note}
只要 \tcode{ranges::swap(E1, E2)} 是一个有效的表达式,它就会
交换 \tcode{E1} 和 \tcode{E2} 所表示的值,并且类型为 \keyword{void}。
\end{note}

\begin{itemdecl}
template<class T>
  concept @\deflibconcept{swappable}@ = requires(T& a, T& b) { ranges::swap(a, b); };
\end{itemdecl}

\begin{itemdecl}
template<class T, class U>
  concept @\deflibconcept{swappable_with}@ =
    @\libconcept{common_reference_with}@<T, U> &&
    requires(T&& t, U&& u) {
      ranges::swap(std::forward<T>(t), std::forward<T>(t));
      ranges::swap(std::forward<U>(u), std::forward<U>(u));
      ranges::swap(std::forward<T>(t), std::forward<U>(u));
      ranges::swap(std::forward<U>(u), std::forward<T>(t));
    };
\end{itemdecl}

\pnum
\begin{note}
\libconcept{swappable} 和 \libconcept{swappable_with}
概念的语义完全由 \tcode{ranges::swap} 定制点对象定义。
\end{note}

\pnum
\begin{example}
用户代码可以确保在各种条件下,
在适当的上下文中执行 \tcode{swap} 调用的评估,如下所示:
\begin{codeblock}
#include <cassert>
#include <concepts>
#include <utility>

namespace ranges = std::ranges;

template<class T, std::@\libconcept{swappable_with}@<T> U>
void value_swap(T&& t, U&& u) {
  ranges::swap(std::forward<T>(t), std::forward<U>(u));
}

template<std::@\libconcept{swappable}@ T>
void lv_swap(T& t1, T& t2) {
  ranges::swap(t1, t2);
}

namespace N {
  struct A { int m; };
  struct Proxy {
    A* a;
    Proxy(A& a) : a{&a} {}
    friend void swap(Proxy x, Proxy y) {
      ranges::swap(*x.a, *y.a);
    }
  };
  Proxy proxy(A& a) { return Proxy{ a }; }
}

int main() {
  int i = 1, j = 2;
  lv_swap(i, j);
  assert(i == 2 && j == 1);

  N::A a1 = { 5 }, a2 = { -5 };
  value_swap(a1, proxy(a2));
  assert(a1.m == -5 && a2.m == 5);
}
\end{codeblock}
\end{example}

\rSec2[concept.destructible]{概念 \cname{destructible}}

\pnum
\libconcept{destructible} 概念指定了所有类型的属性,
其的实例可以在其生命周期结束时被销毁,或者引用
类型。

\begin{itemdecl}
template<class T>
concept @\deflibconcept{destructible}@ = is_nothrow_destructible_v<T>;
\end{itemdecl}

\begin{itemdescr}
\pnum
\begin{note}
与 \oldconcept{Destructible} 要求~(\tref{cpp17.destructible}) 不同,此
概念禁止可能是抛出的析构函数,即使析构函数的特定
调用实际上不会抛出。
\end{note}
\end{itemdescr}

\rSec2[concept.constructible]{概念 \cname{constructible_from}}

\pnum
\libconcept{constructible_from} 概念约束了具有特定参数类型集合的
给定类型的变量的初始化。

\begin{itemdecl}
template<class T, class... Args>
concept @\deflibconcept{constructible_from}@ = @\libconcept{destructible}@<T> && is_constructible_v<T, Args...>;
\end{itemdecl}

\rSec2[concept.default.init]{概念 \cname{default_initializable}}

\begin{itemdecl}
template<class T>
constexpr bool @\exposid{is-default-initializable}@ = @\seebelow@;         // \expos

template<class T>
concept @\deflibconcept{default_initializable}@ = @\libconcept{constructible_from}@<T> &&
                                requires { T{}; } &&
                                @\exposid{is-default-initializable}@<T>;
\end{itemdecl}

\begin{itemdescr}
\pnum
对于类型 \tcode{T},\tcode{\exposid{is-default-initializable}<T>} 为 \tcode{true}
当且仅当变量定义
\begin{codeblock}
T t;
\end{codeblock}
对于某个虚构的变量 \tcode{t} 是良构的;
否则为 \tcode{false}。
访问检查的执行就像在与 \tcode{T} 无关的上下文中一样。
仅考虑变量初始化的直接上下文的有效性。
\end{itemdescr}

\rSec2[concept.moveconstructible]{概念 \cname{move_constructible}}

\begin{itemdecl}
template<class T>
concept @\deflibconcept{move_constructible}@ = @\libconcept{constructible_from}@<T, T> && @\libconcept{convertible_to}@<T, T>;
\end{itemdecl}

\begin{itemdescr}
\pnum
如果 \tcode{T} 是对象类型,则令 \tcode{rv} 为类型
\tcode{T} 的右值,并且 \tcode{u2} 为等于
\tcode{rv} 的类型 \tcode{T} 的不同对象。\tcode{T} 建模 \libconcept{move_constructible} 仅当

\begin{itemize}
\item 在定义 \tcode{T u = rv;} 之后,\tcode{u} 等于 \tcode{u2}。

\item \tcode{T(rv)} 等于 \tcode{u2}。

\item 如果 \tcode{T} 不是 \keyword{const},则 \tcode{rv} 的结果状态是有效的
但未指定的\iref{lib.types.movedfrom};否则,它是不变的。
\end{itemize}
\end{itemdescr}

\rSec2[concept.copyconstructible]{概念 \cname{copy_constructible}}

\begin{itemdecl}
template<class T>
concept @\deflibconcept{copy_constructible}@ =
  @\libconcept{move_constructible}@<T> &&
  @\libconcept{constructible_from}@<T, T&> && @\libconcept{convertible_to}@<T&, T> &&
  @\libconcept{constructible_from}@<T, const T&> && @\libconcept{convertible_to}@<const T&, T> &&
  @\libconcept{constructible_from}@<T, const T> && @\libconcept{convertible_to}@<const T, T>;
\end{itemdecl}

\begin{itemdescr}
\pnum
如果 \tcode{T} 是对象类型,则令 \tcode{v} 为类型
\tcode{T} 或 \tcode{\keyword{const} T} 的左值,或类型 \tcode{\keyword{const} T} 的右值。
\tcode{T} 建模 \libconcept{copy_constructible} 仅当

\begin{itemize}
\item 在定义 \tcode{T u = v;} 之后,
\tcode{u} 等于 \tcode{v}\iref{concepts.equality} 并且
\tcode{v} 未被修改。

\item \tcode{T(v)} 等于 \tcode{v} 并且不修改 \tcode{v}。
\end{itemize}

\end{itemdescr}

\rSec1[concepts.compare]{比较概念}

\rSec2[concepts.compare.general]{概述}

\pnum
子条款 \ref{concepts.compare} 描述了在可能不同的对象类型的值上建立关系和排序的概念。

\pnum
给定一个表达式 \tcode{E} 和一个类型 \tcode{C},
令 \tcode{\exposid{CONVERT_TO_LVALUE}<C>(E)} 为:
\begin{itemize}
\item
如果 \tcode{static_cast<const C\&>(as_const(E))} 是一个有效的表达式,则为它,以及
\item
否则为 \tcode{static_cast<const C\&>(std::move(E))}。
\end{itemize}

\rSec2[concept.booleantestable]{布尔可测试性}

\pnum
仅用于说明的 \exposconcept{boolean-testable} 概念
指定了对可转换为 \tcode{bool} 并且
逻辑运算符\iref{expr.log.and,expr.log.or,expr.unary.op}
具有常规语义的表达式的要求。

\begin{itemdecl}
template<class T>
concept @\defexposconcept{boolean-testable-impl}@ = @\libconcept{convertible_to}@<T, bool>;  // \expos
\end{itemdecl}

\pnum
令 \tcode{e} 为一个表达式,使得
\tcode{decltype((e))} 是 \tcode{T}。
\tcode{T} 建模 \exposconcept{boolean-testable-impl} 仅当

\begin{itemize}
\item
\tcode{remove_cvref_t<T>} 不是类类型,或者
在 \tcode{remove_cvref_t<T>} 的作用域中搜索名称 \tcode{operator\&\&} 和 \tcode{operator||}
未找到任何内容;以及

\item
以 \tcode{T} 作为唯一参数类型
对名称 \tcode{operator\&\&} 和 \tcode{operator||} 进行依赖于实参的查找\iref{basic.lookup.argdep}
找不到不合格的声明(定义如下)。
\end{itemize}

\pnum
\defnadj{不合格}{形参}
是一个函数形参,其声明的类型 \tcode{P}

\begin{itemize}
\item
不依赖于模板形参,并且
存在从 \tcode{e} 到 \tcode{P} 的隐式转换序列\iref{over.best.ics};或者

\item
依赖于一个或多个模板形参,并且满足以下条件之一
\begin{itemize}
\item
\tcode{P} 不包含参与模板实参推导\iref{temp.deduct.type} 的模板形参,或者
\item
使用函数调用中推导模板实参的规则\iref{temp.deduct.call} 进行的模板实参推导和
\tcode{e} 作为实参成功。
\end{itemize}
\end{itemize}

\pnum
\indextext{template!function!key parameter of}%
函数模板 \tcode{D} 的 \defnadj{关键}{形参}
是类型为 \cv{} \tcode{X} 或其引用的函数形参,
其中 \tcode{X} 命名一个类模板的特化,该类模板具有与 \tcode{D} 相同的最内层封闭非内联命名空间,并且
\tcode{X} 包含至少一个参与模板实参推导的模板形参。
\begin{example}
在
\begin{codeblock}
namespace Z {
template<class> struct C {};
template<class T>
  void operator&&(C<T> x, T y);
template<class T>
  void operator||(C<type_identity_t<T>> x, T y);
}
\end{codeblock}
\tcode{Z::operator\&\&} 的声明
包含一个关键形参 \tcode{C<T> x},并且
\tcode{Z::operator||} 的声明
不包含关键形参。
\end{example}

\pnum
\defnadj{不合格}{声明} 是

\begin{itemize}
\item
包含至少一个不合格形参的(非模板)函数声明;或者

\item
包含至少一个不合格形参的函数模板声明,其中
\begin{itemize}
\item 至少一个不合格形参是关键形参;或者
\item 声明不包含关键形参;或者
\item 声明声明了一个未绑定任何名称的函数模板\iref{dcl.meaning}。
\end{itemize}
\end{itemize}

\pnum
\begin{note}
目的是确保
给定两个类型 \tcode{T1} 和 \tcode{T2}
每个都建模 \exposconcept{boolean-testable-impl},
表达式 \tcode{declval<T1>() \&\& declval<T2>()} 和
\tcode{declval<T1>() || declval<T2>()} 中的 \tcode{\&\&} 和 \tcode{||} 运算符
解析为相应的内置运算符。
\end{note}

\begin{itemdecl}
template<class T>
concept @\defexposconcept{boolean-testable}@ =                // \expos
  @\exposconcept{boolean-testable-impl}@<T> && requires(T&& t) {
    { !std::forward<T>(t) } -> @\exposconcept{boolean-testable-impl}@;
  };
\end{itemdecl}

\pnum
令 \tcode{e} 为一个表达式,使得
\tcode{decltype((e))} 是 \tcode{T}。
\tcode{T} 建模 \exposconcept{boolean-testable} 仅当
\tcode{bool(e) == !bool(!e)}。

\pnum
\begin{example}
类型
\tcode{bool}、
\tcode{true_type}\iref{meta.type.synop}、
\tcode{int*} 和
\tcode{bitset<N>::reference}\iref{template.bitset}
建模 \exposconcept{boolean-testable}。
\end{example}

\rSec2[concept.comparisoncommontype]{比较公共类型}

\begin{itemdecl}
template<class T, class U, class C = common_reference_t<const T&, const U&>>
concept @\defexposconcept{comparison-common-type-with-impl}@ =   // \expos
  @\libconcept{same_as}@<common_reference_t<const T&, const U&>,
          common_reference_t<const U&, const T&>> &&
  requires {
    requires @\libconcept{convertible_to}@<const T&, const C&> || @\libconcept{convertible_to}@<T, const C&>;
    requires @\libconcept{convertible_to}@<const U&, const C&> || @\libconcept{convertible_to}@<U, const C&>;
  };

template<class T, class U>
concept @\defexposconcept{comparison-common-type-with}@ =   // \expos
  @\exposconcept{comparison-common-type-with-impl}@<remove_cvref_t<T>, remove_cvref_t<U>>;
\end{itemdecl}

\pnum
令 \tcode{C} 为 \tcode{common_reference_t<const T\&, const U\&>}。
令 \tcode{t1} 和 \tcode{t2} 为等式保持表达式
作为类型 \tcode{remove_cvref_t<T>} 的左值,并且
令 \tcode{u1} 和 \tcode{u2} 为等式保持表达式
作为类型 \tcode{remove_cvref_t<U>} 的左值。
\tcode{T} 和 \tcode{U} 建模
\tcode{\exposconcept{comparison-common-type-with}<T, U>} 仅当
\begin{itemize}
\item
\tcode{\exposid{CONVERT_TO_LVALUE}<C>(t1)} 等于
\tcode{\exposid{CONVERT_TO_LVALUE}<C>(t2)}
当且仅当 \tcode{t1} 等于 \tcode{t2},并且
\item
\tcode{\exposid{CONVERT_TO_LVALUE}<C>(u1)} 等于
\tcode{\exposid{CONVERT_TO_LVALUE}<C>(u2)}
当且仅当 \tcode{u1} 等于 \tcode{u2}
\end{itemize}

\rSec2[concept.equalitycomparable]{概念 \cname{equality_comparable}}

\begin{itemdecl}
template<class T, class U>
  concept @\defexposconcept{weakly-equality-comparable-with}@ = // \expos
    requires(const remove_reference_t<T>& t,
             const remove_reference_t<U>& u) {
      { t == u } -> @\exposconcept{boolean-testable}@;
      { t != u } -> @\exposconcept{boolean-testable}@;
      { u == t } -> @\exposconcept{boolean-testable}@;
      { u != t } -> @\exposconcept{boolean-testable}@;
    };
\end{itemdecl}

\begin{itemdescr}
\pnum
给定类型 \tcode{T} 和 \tcode{U},
令 \tcode{t} 和 \tcode{u} 分别为类型
\tcode{const remove_reference_t<T>} 和
\tcode{const remove_reference_t<U>} 的左值。
\tcode{T} 和 \tcode{U} 建模
\tcode{\exposconcept{weakly-equality-comparable-with}<T, U>} 仅当
\begin{itemize}
\item \tcode{t == u}、\tcode{u == t}、\tcode{t != u} 和 \tcode{u != t}
      具有相同的域。
\item \tcode{bool(u == t) ==  bool(t == u)}。
\item \tcode{bool(t != u) == !bool(t == u)}。
\item \tcode{bool(u != t) ==  bool(t != u)}。
\end{itemize}
\end{itemdescr}

\begin{itemdecl}
template<class T>
  concept @\deflibconcept{equality_comparable}@ = @\exposconcept{weakly-equality-comparable-with}@<T, T>;
\end{itemdecl}

\begin{itemdescr}
\pnum
令 \tcode{a} 和 \tcode{b} 为类型 \tcode{T} 的对象。
当 \tcode{a} 等于
\tcode{b}\iref{concepts.equality} 时,\tcode{T} 建模 \libconcept{equality_comparable} 仅当 \tcode{bool(a == b)} 为 \tcode{true},否则为 \tcode{false}。

\pnum
\begin{note}
表达式 \tcode{a == b} 是等式保持的要求
暗示 \tcode{==} 是传递的和对称的。
\end{note}
\end{itemdescr}

\begin{itemdecl}
template<class T, class U>
  concept @\deflibconcept{equality_comparable_with}@ =
    @\libconcept{equality_comparable}@<T> && @\libconcept{equality_comparable}@<U> &&
    @\exposconcept{comparison-common-type-with}@<T, U> &&
    @\libconcept{equality_comparable}@<
      common_reference_t<
        const remove_reference_t<T>&,
        const remove_reference_t<U>&>> &&
    @\exposconcept{weakly-equality-comparable-with}@<T, U>;
\end{itemdecl}

\begin{itemdescr}
\pnum
给定类型 \tcode{T} 和 \tcode{U},
令 \tcode{t} 和 \tcode{t2} 为左值
分别表示类型 \tcode{const remove_reference_t<T>} 和
\tcode{remove_cvref_t<T>} 的不同的相等对象,
令 \tcode{u} 和 \tcode{u2} 为左值
分别表示类型 \tcode{const remove_reference_t<U>} 和
\tcode{remove_cvref_t<U>} 的不同的相等对象,并且
令 \tcode{C} 为:
\begin{codeblock}
common_reference_t<const remove_reference_t<T>&, const remove_reference_t<U>&>
\end{codeblock}
\tcode{T} 和 \tcode{U} 建模
\tcode{\libconcept{equality_comparable_with}<T, U>} 仅当
\begin{codeblock}
bool(t == u) == bool(@\exposid{CONVERT_TO_LVALUE}@<C>(t2) == @\exposid{CONVERT_TO_LVALUE}@<C>(u2))
\end{codeblock}
\end{itemdescr}

\rSec2[concept.totallyordered]{概念 \cname{totally_ordered}}

\begin{itemdecl}
template<class T>
  concept @\deflibconcept{totally_ordered}@ =
    @\libconcept{equality_comparable}@<T> && @\exposconcept{partially-ordered-with}@<T, T>;
\end{itemdecl}

\begin{itemdescr}
\pnum
给定一个类型 \tcode{T},令 \tcode{a}、\tcode{b} 和 \tcode{c} 为
类型 \tcode{const remove_reference_t<T>} 的左值。
\tcode{T} 建模 \libconcept{totally_ordered} 仅当

\begin{itemize}
\item \tcode{bool(a < b)}、\tcode{bool(a > b)} 或
      \tcode{bool(a == b)} 中恰好有一个为 \tcode{true}。
\item 如果 \tcode{bool(a < b)} 且 \tcode{bool(b < c)},则
      \tcode{bool(a < c)}。
\item \tcode{bool(a <= b) == !bool(b < a)}。
\item \tcode{bool(a >= b) == !bool(a < b)}。
\end{itemize}

\end{itemdescr}

\begin{itemdecl}
template<class T, class U>
  concept @\deflibconcept{totally_ordered_with}@ =
    @\libconcept{totally_ordered}@<T> && @\libconcept{totally_ordered}@<U> &&
    @\libconcept{equality_comparable_with}@<T, U> &&
    @\libconcept{totally_ordered}@<
      common_reference_t<
        const remove_reference_t<T>&,
        const remove_reference_t<U>&>> &&
    @\exposconcept{partially-ordered-with}@<T, U>;
\end{itemdecl}

\begin{itemdescr}
\pnum
给定类型 \tcode{T} 和 \tcode{U},
令 \tcode{t} 和 \tcode{t2} 为左值
分别表示类型 \tcode{const remove_reference_t<T>} 和
\tcode{remove_cvref_t<T>} 的不同的相等对象,
令 \tcode{u} 和 \tcode{u2} 为左值
分别表示类型 \tcode{const remove_reference_t<U>} 和
\tcode{remove_cvref_t<U>} 的不同的相等对象,并且
令 \tcode{C} 为:
\begin{codeblock}
common_reference_t<const remove_reference_t<T>&, const remove_reference_t<U>&>
\end{codeblock}
\tcode{T} 和 \tcode{U} 建模
\tcode{\libconcept{totally_ordered_with}<T, U>} 仅当
\begin{itemize}
\item \tcode{bool(t <  u) == bool(\exposid{CONVERT_TO_LVALUE}<C>(t2) <  \exposid{CONVERT_TO_LVALUE}<C>(u2))}.
\item \tcode{bool(t >  u) == bool(\exposid{CONVERT_TO_LVALUE}<C>(t2) >  \exposid{CONVERT_TO_LVALUE}<C>(u2))}.
\item \tcode{bool(t <= u) == bool(\exposid{CONVERT_TO_LVALUE}<C>(t2) <= \exposid{CONVERT_TO_LVALUE}<C>(u2))}.
\item \tcode{bool(t >= u) == bool(\exposid{CONVERT_TO_LVALUE}<C>(t2) >= \exposid{CONVERT_TO_LVALUE}<C>(u2))}.
\item \tcode{bool(u <  t) == bool(\exposid{CONVERT_TO_LVALUE}<C>(u2) <  \exposid{CONVERT_TO_LVALUE}<C>(t2))}.
\item \tcode{bool(u >  t) == bool(\exposid{CONVERT_TO_LVALUE}<C>(u2) >  \exposid{CONVERT_TO_LVALUE}<C>(t2))}.
\item \tcode{bool(u <= t) == bool(\exposid{CONVERT_TO_LVALUE}<C>(u2) <= \exposid{CONVERT_TO_LVALUE}<C>(t2))}.
\item \tcode{bool(u >= t) == bool(\exposid{CONVERT_TO_LVALUE}<C>(u2) >= \exposid{CONVERT_TO_LVALUE}<C>(t2))}.
\end{itemize}
\end{itemdescr}

\rSec1[concepts.object]{对象概念}

\pnum
本子条款描述了指定库所基于的面向值编程风格的基础的概念。

\begin{itemdecl}
template<class T>
  concept @\deflibconcept{movable}@ = is_object_v<T> && @\libconcept{move_constructible}@<T> &&
                    @\libconcept{assignable_from}@<T&, T> && @\libconcept{swappable}@<T>;
template<class T>
  concept @\deflibconcept{copyable}@ = @\libconcept{copy_constructible}@<T> && @\libconcept{movable}@<T> && @\libconcept{assignable_from}@<T&, T&> &&
                     @\libconcept{assignable_from}@<T&, const T&> && @\libconcept{assignable_from}@<T&, const T>;
template<class T>
  concept @\deflibconcept{semiregular}@ = @\libconcept{copyable}@<T> && @\libconcept{default_initializable}@<T>;
template<class T>
  concept @\deflibconcept{regular}@ = @\libconcept{semiregular}@<T> && @\libconcept{equality_comparable}@<T>;
\end{itemdecl}

\begin{itemdescr}
\pnum
\begin{note}
\libconcept{semiregular} 概念由行为类似于
像 \tcode{int} 这样的基本类型,只是它们不需要
与 \tcode{==} 进行比较的类型建模。
\end{note}

\pnum
\begin{note}
\libconcept{regular} 概念由行为类似于
像 \tcode{int} 这样的基本类型并且可以与 \tcode{==} 进行比较的类型建模。
\end{note}
\end{itemdescr}

\rSec1[concepts.callable]{可调用概念}

\rSec2[concepts.callable.general]{概述}

\pnum
\ref{concepts.callable} 中的概念描述了对函数
对象\iref{function.objects} 及其参数的要求。

\rSec2[concept.invocable]{概念 \cname{invocable}}

\pnum
\libconcept{invocable} 概念指定了可调用
类型\iref{func.def} \tcode{F} 和一组参数类型 \tcode{Args...} 之间的关系,它们
可以通过库函数 \tcode{invoke}\iref{func.invoke} 进行评估。

\begin{itemdecl}
template<class F, class... Args>
  concept @\deflibconcept{invocable}@ = requires(F&& f, Args&&... args) {
    invoke(std::forward<F>(f), std::forward<Args>(args)...); // not required to be equality-preserving
  };
\end{itemdecl}

\begin{itemdescr}
\pnum
\begin{example}
生成随机数的函数可以建模 \libconcept{invocable},
因为 \tcode{invoke} 函数调用表达式不需要
等式保持\iref{concepts.equality}。
\end{example}
\end{itemdescr}

\rSec2[concept.regularinvocable]{概念 \cname{regular_invocable}}

\begin{itemdecl}
template<class F, class... Args>
  concept @\deflibconcept{regular_invocable}@ = @\libconcept{invocable}@<F, Args...>;
\end{itemdecl}

\begin{itemdescr}
\pnum
\tcode{invoke} 函数调用表达式应为
等式保持\iref{concepts.equality} 并且
不应修改函数对象或参数。
\begin{note}
此要求取代了 \libconcept{invocable} 定义中的注释。
\end{note}

\pnum
\begin{example}
随机数生成器不建模 \libconcept{regular_invocable}。
\end{example}

\pnum
\begin{note}
\libconcept{invocable} 和 \libconcept{regular_invocable} 之间的区别
纯粹是语义上的。
\end{note}
\end{itemdescr}

\rSec2[concept.predicate]{概念 \cname{predicate}}

\begin{itemdecl}
template<class F, class... Args>
  concept @\deflibconcept{predicate}@ =
    @\libconcept{regular_invocable}@<F, Args...> && @\exposconcept{boolean-testable}@<invoke_result_t<F, Args...>>;
\end{itemdecl}

\rSec2[concept.relation]{概念 \cname{relation}}

\begin{itemdecl}
template<class R, class T, class U>
  concept @\deflibconcept{relation}@ =
    @\libconcept{predicate}@<R, T, T> && @\libconcept{predicate}@<R, U, U> &&
    @\libconcept{predicate}@<R, T, U> && @\libconcept{predicate}@<R, U, T>;
\end{itemdecl}

\rSec2[concept.equiv]{概念 \cname{equivalence_relation}}

\begin{itemdecl}
template<class R, class T, class U>
  concept @\deflibconcept{equivalence_relation}@ = @\libconcept{relation}@<R, T, U>;
\end{itemdecl}

\begin{itemdescr}
\pnum
\libconcept{relation} 建模 \libconcept{equivalence_relation} 仅当
它对其参数施加等价关系。
\end{itemdescr}

\rSec2[concept.strictweakorder]{概念 \cname{strict_weak_order}}

\begin{itemdecl}
template<class R, class T, class U>
  concept @\deflibconcept{strict_weak_order}@ = @\libconcept{relation}@<R, T, U>;
\end{itemdecl}

\begin{itemdescr}
\pnum
\libconcept{relation} 建模 \libconcept{strict_weak_order} 仅当
它对其参数施加\term{严格弱序}。

\pnum
术语
\term{严格}
指的是
非自反关系的要求(对于所有 \tcode{x},\tcode{!comp(x, x)}),
以及术语
\term{弱}
指的是不如全序的要求那么强,
但比偏序的要求更强。
如果我们将
\tcode{equiv(a, b)}
定义为
\tcode{!comp(a, b) \&\& !comp(b, a)},
那么要求是
\tcode{comp}
和
\tcode{equiv}
都是传递关系:

\begin{itemize}
\item
\tcode{comp(a, b) \&\& comp(b, c)}
暗示
\tcode{comp(a, c)}
\item
\tcode{equiv(a, b) \&\& equiv(b, c)}
暗示
\tcode{equiv(a, c)}
\end{itemize}

\pnum
\begin{note}
在这些条件下,可以证明
\begin{itemize}
\item
\tcode{equiv}
是一个等价关系,
\item
\tcode{comp}
在由 \tcode{equiv} 确定的等价类上导出一个明确定义的关系,并且
\item
导出的关系是一个严格全序。
\end{itemize}
\end{note}
\end{itemdescr}
