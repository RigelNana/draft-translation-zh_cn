%!TEX root = std.tex
\normannex{depr}{兼容性特性}

\rSec1[depr.general]{概述}

\pnum
本附件描述了本文档中
为与现有实现兼容而指定的功能。

\pnum
这些是不推荐使用的功能,其中
\term{deprecated}
定义为:
对于 \Cpp{} 的当前修订版是规范的,
但已被确定为从未来修订版中删除的候选对象。
实现可以使用
\tcode{deprecated} 属性\iref{dcl.attr.deprecated} 声明本条款中描述的库名称和实体。

\rSec1[depr.local]{TU-local 实体的非局部使用}

\pnum
作为暴露\iref{basic.link} 的非 TU-local 实体的声明
是不推荐使用的。
\begin{note}
在可导入模块单元中的此类声明是病态的。
\end{note}
\begin{example}
\begin{codeblock}
namespace {
  struct A {
    void f() {}
  };
}
A h();                          // 不推荐使用:不是内部链接
inline void g() {A().f();}      // 不推荐使用:内联且不是内部链接
\end{codeblock}
\end{example}

\rSec1[depr.capture.this]{通过引用隐式捕获 \tcode{*this}}

\pnum
为了与 \Cpp{} 的先前修订版兼容,
带有 \grammarterm{capture-default}
\tcode{=}\iref{expr.prim.lambda.capture} 的 \grammarterm{lambda-expression} 可以通过引用隐式捕获
\tcode{*this}。
\begin{example}
\begin{codeblock}
struct X {
  int x;
  void foo(int n) {
    auto f = [=]() { x = n; };          // 不推荐使用:\tcode{x} 表示 \tcode{this->x},而不是其副本
    auto g = [=, this]() { x = n; };    // 推荐的替换
  }
};
\end{codeblock}
\end{example}

\rSec1[depr.volatile.type]{不推荐使用的 \tcode{volatile} 类型}

\pnum
volatile 限定的算术类型和指针类型的
后缀 \tcode{++} 和 \tcode{--} 表达式\iref{expr.post.incr} 以及
前缀 \tcode{++} 和 \tcode{--} 表达式\iref{expr.pre.incr}
是不推荐使用的。

\begin{example}
\begin{codeblock}
volatile int velociraptor;
++velociraptor;                     // 不推荐使用
\end{codeblock}
\end{example}


\pnum
某些左操作数是 volatile 限定的非类类型的赋值
是不推荐使用的;请参见~\ref{expr.ass}。

\begin{example}
\begin{codeblock}
int neck, tail;
volatile int brachiosaur;
brachiosaur = neck;                 // OK
tail = brachiosaur;                 // OK
tail = brachiosaur = neck;          // 不推荐使用
brachiosaur += neck;                // OK
\end{codeblock}
\end{example}


\pnum
带有 volatile 限定类型的参数或
带有 volatile 限定返回类型的函数类型\iref{dcl.fct}
是不推荐使用的。

\begin{example}
\begin{codeblock}
volatile struct amber jurassic();                               // 不推荐使用
void trex(volatile short left_arm, volatile short right_arm);   // 不推荐使用
void fly(volatile struct pterosaur* pteranodon);                // OK
\end{codeblock}
\end{example}


\pnum
volatile 限定类型的结构化绑定\iref{dcl.struct.bind}
是不推荐使用的。

\begin{example}
\begin{codeblock}
struct linhenykus { short forelimb; };
void park(linhenykus alvarezsauroid) {
  volatile auto [what_is_this] = alvarezsauroid;                // 不推荐使用
  // ...
}
\end{codeblock}
\end{example}

\rSec1[depr.ellipsis.comma]{非逗号分隔的省略号参数}

形式为
\grammarterm{parameter-declaration-list} \tcode{...} 的 \grammarterm{parameter-declaration-clause}
是不推荐使用的\iref{dcl.fct}。
\begin{example}
\begin{codeblock}
void f(int...);         // 不推荐使用
void g(auto...);        // OK,声明函数参数包
void h(auto......);     // 不推荐使用
\end{codeblock}
\end{example}

\rSec1[depr.impldec]{复制函数的隐式声明}

\pnum
如果类具有
用户声明的复制赋值运算符或
用户声明的析构函数\iref{class.dtor},则复制构造函数\iref{class.copy.ctor}
的隐式定义为默认定义是不推荐使用的。
如果类具有
用户声明的复制构造函数或
用户声明的析构函数,则复制赋值运算符\iref{class.copy.assign}
的隐式定义为默认定义是不推荐使用的。
\Cpp{} 的未来版本可能会指定
删除这些隐式定义\iref{dcl.fct.def.delete}。

\rSec1[depr.static.constexpr]{\tcode{static constexpr} 数据成员的重新声明}

\pnum
为了与 \Cpp{} 的先前修订版兼容,\keyword{constexpr}
静态数据成员可以在类外部冗余地重新声明,而无需
初始化器\iref{basic.def,class.static.data}。
此用法是不推荐使用的。
\begin{example}
\begin{codeblock}
struct A {
  static constexpr int n = 5;   // 定义(\CppXIV{} 中的声明)
};

constexpr int A::n;             // 冗余声明(\CppXIV{} 中的定义)
\end{codeblock}
\end{example}

\rSec1[depr.lit]{使用标识符的文字运算符函数声明}

\pnum
以下形式的 \grammarterm{literal-operator-id}\iref{over.literal}
\begin{codeblock}
operator @\grammarterm{unevaluated-string}@ @\grammarterm{identifier}@
\end{codeblock}
是不推荐使用的。

\rSec1[depr.template.template]{限定名称前的 \tcode{template} 关键字}

\pnum
在没有模板参数列表的类或别名模板的
限定名称之前使用关键字 \keyword{template}
是不推荐使用的\iref{temp.names}。

\rSec1[depr.numeric.limits.has.denorm]{\tcode{numeric_limits} 中的 \tcode{has_denorm} 成员}

\pnum
除了 \libheaderref{limits} 中指定的类型外,还定义了以下类型:
\begin{codeblock}
namespace std {
  enum @\libglobal{float_denorm_style}@ {
    @\libmember{denorm_indeterminate}{float_denorm_style}@ = -1,
    @\libmember{denorm_absent}{float_denorm_style}@ = 0,
    @\libmember{denorm_present}{float_denorm_style}@ = 1
  };
}
\end{codeblock}

\pnum
\indextext{denormalized value|see{number, subnormal}}%
\indextext{value!denormalized|see{number, subnormal}}%
\indextext{subnormal number|see{number, subnormal}}%
\indextext{number!subnormal}%
除了 \ref{numeric.limits.general} 中指定的成员外,还定义了以下成员:
\begin{codeblock}
static constexpr float_denorm_style has_denorm = denorm_absent;
static constexpr bool has_denorm_loss = false;
\end{codeblock}

\pnum
\tcode{numeric_limits} 特化版本的 \tcode{has_denorm} 和 \tcode{has_denorm_loss} 的值是未指定的。

\pnum
除了 \ref{numeric.special} 中指定的成员外,还定义了特化版本 \tcode{numeric_limits<bool>} 的以下成员:
\indexlibrarymember{float_denorm_style}{numeric_limits}%
\indexlibrarymember{has_denorm_loss}{numeric_limits}%
\begin{codeblock}
static constexpr float_denorm_style has_denorm = denorm_absent;
static constexpr bool has_denorm_loss = false;
\end{codeblock}

\rSec1[depr.c.macros]{不推荐使用的 C 宏}

\pnum
头文件 \libheaderref{stdalign.h} 具有以下宏:
\begin{codeblock}
#define @\libxmacro{alignas_is_defined}@ 1
#define @\libxmacro{alignof_is_defined}@ 1
\end{codeblock}

\pnum
头文件 \libheaderref{stdbool.h} 具有以下宏:
\begin{codeblock}
#define @\libxmacro{bool_true_false_are_defined}@ 1
\end{codeblock}

\rSec1[depr.cerrno]{不推荐使用的错误编号}

\pnum
头文件 \libheaderref{cerrno} 具有以下附加宏:

\begin{codeblock}
#define @\libmacro{ENODATA}@ @\seebelow@
#define @\libmacro{ENOSR}@ @\seebelow@
#define @\libmacro{ENOSTR}@ @\seebelow@
#define @\libmacro{ETIME}@ @\seebelow@
\end{codeblock}

\pnum
这些宏的含义由 POSIX 标准定义。

\pnum
除了 \ref{system.error.syn} 中指定的 \tcode{enum errc} 枚举器外,还定义了以下枚举器:

\begin{codeblock}
@\libmember{no_message_available}{errc}@,               // \tcode{ENODATA}
@\libmember{no_stream_resources}{errc}@,                // \tcode{ENOSR}
@\libmember{not_a_stream}{errc}@,                       // \tcode{ENOSTR}
@\libmember{stream_timeout}{errc}@,                     // \tcode{ETIME}
\end{codeblock}

\pnum
上述每个 \tcode{enum errc} 枚举器的值
与上述概要中所示的 \libheader{cerrno} 宏的值相同。

\rSec1[depr.meta.types]{不推荐使用的类型特征}

\pnum
头文件 \libheaderrefx{type_traits}{meta.type.synop}
具有以下附加内容:

\begin{codeblock}
namespace std {
  template<class T> struct is_trivial;
  template<class T> constexpr bool is_trivial_v = is_trivial<T>::value;
  template<class T> struct is_pod;
  template<class T> constexpr bool is_pod_v = is_pod<T>::value;
  template<size_t Len, size_t Align = @\exposid{default-alignment}@> // \seebelow
    struct aligned_storage;
  template<size_t Len, size_t Align = @\exposid{default-alignment}@> // \seebelow
    using @\libglobal{aligned_storage_t}@ = typename aligned_storage<Len, Align>::type;
  template<size_t Len, class... Types>
    struct aligned_union;
  template<size_t Len, class... Types>
    using @\libglobal{aligned_union_t}@ = typename aligned_union<Len, Types...>::type;
}
\end{codeblock}

\pnum
添加了为此子条款中定义的任何模板的特化的程序的行为是未定义的,
除非相应模板的规范明确允许。

\pnum
\label{term.trivial.type}%
\defnadj{平凡}{类} 是一个可平凡复制的类,
并且具有一个或多个合格的默认构造函数,所有这些构造函数都是平凡的。
\begin{note}
特别是,
平凡类没有虚函数或虚基类。
\end{note}
\defnadj{平凡}{类型} 是标量类型、平凡类、
此类类型的数组或其中一种类型的 cv 限定版本。

\pnum
\indextext{POD}%
\term{POD 类} 是一个既是平凡类又是
标准布局类的类,并且没有类型为非 POD 类
(或其数组)的非静态数据成员。\term{POD 类型} 是标量类型、POD 类、此类
类型的数组或其中一种类型的 cv 限定版本。

\indexlibraryglobal{is_trivial}%
\begin{itemdecl}
template<class T> struct is_trivial;
\end{itemdecl}

\begin{itemdescr}
\pnum
\expects
\tcode{remove_all_extents_t<T>} 应为完整类型或 \cv{} \keyword{void}。

\pnum
\remarks
\tcode{is_trivial<T>} 是一个 \oldconcept{UnaryTypeTrait}\iref{meta.rqmts},
如果 \tcode{T} 是平凡类型,则具有 \tcode{true_type} 的基本特征,
否则具有 \tcode{false_type}。

\pnum
\begin{note}
未指定闭包类型\iref{expr.prim.lambda.closure} 是否为平凡类型。
\end{note}
\end{itemdescr}

\indexlibraryglobal{is_pod}%
\begin{itemdecl}
template<class T> struct is_pod;
\end{itemdecl}

\begin{itemdescr}
\pnum
\expects
\tcode{remove_all_extents_t<T>} 应为完整类型或 \cv{} \keyword{void}。

\pnum
\remarks
\tcode{is_pod<T>} 是一个 \oldconcept{UnaryTypeTrait}\iref{meta.rqmts},
如果 \tcode{T} 是 POD 类型,则具有 \tcode{true_type} 的基本特征,
否则具有 \tcode{false_type}。

\pnum
\begin{note}
未指定闭包类型\iref{expr.prim.lambda.closure} 是否为 POD 类型。
\end{note}
\end{itemdescr}

\indexlibraryglobal{aligned_storage}%
\begin{itemdecl}
template<size_t Len, size_t Align = @\exposid{default-alignment}@>
  struct aligned_storage;
\end{itemdecl}

\begin{itemdescr}
\pnum
\exposid{default-alignment} 的值是
任何大小不大于 \tcode{Len}\iref{basic.types} 的对象类型的最严格对齐要求。

\pnum
\mandates
\tcode{Len} 不为零。
\tcode{Align} 等于某个类型 \tcode{T} 的 \tcode{alignof(T)} 或
等于 \exposid{default-alignment}。

\pnum
成员 typedef \tcode{type} 表示一个平凡的标准布局类型,
适用于作为任何对象的未初始化存储,
其大小最大为 \tcode{Len} 并且
其对齐方式是 \tcode{Align} 的约数。

\pnum
\begin{note}
可以使用声明为 \tcode{alignas(Align)} 的数组 \tcode{std::byte[Len]} 来替换 \tcode{aligned_storage<Len, Align>::type} 的使用。
\end{note}

\pnum
\begin{note}
典型的实现会将 \tcode{aligned_storage} 定义为:
\begin{codeblock}
template<size_t Len, size_t Alignment>
struct aligned_storage {
  typedef struct {
    alignas(Alignment) unsigned char __data[Len];
  } type;
};
\end{codeblock}
\end{note}

\end{itemdescr}

\indexlibraryglobal{aligned_union}%
\begin{itemdecl}
template<size_t Len, class... Types>
  struct aligned_union;
\end{itemdecl}

\begin{itemdescr}
\pnum
\mandates
至少提供一种类型。
模板参数包 \tcode{Types} 中的每个类型
都是一个完整的对象类型。

\pnum
成员 typedef \tcode{type} 表示一个平凡的标准布局类型,
适用于作为任何对象的未初始化存储,
其类型列在 \tcode{Types} 中;
其大小应至少为 \tcode{Len}。
静态成员 \tcode{alignment_value}
是类型为 \tcode{size_t} 的整数常量,
其值是 \tcode{Types} 中列出的所有类型的最严格对齐方式。
\end{itemdescr}

\rSec1[depr.relops]{关系运算符}%
\indexlibraryglobal{rel_ops}%

\pnum
头文件 \libheaderref{utility} 具有以下附加内容:

\begin{codeblock}
namespace std::rel_ops {
  template<class T> bool operator!=(const T&, const T&);
  template<class T> bool operator> (const T&, const T&);
  template<class T> bool operator<=(const T&, const T&);
  template<class T> bool operator>=(const T&, const T&);
}
\end{codeblock}

\pnum
为了避免从 \tcode{operator==} 中冗余定义 \tcode{operator!=}
以及从 \tcode{operator<} 中冗余定义运算符 \tcode{>}、\tcode{<=} 和 \tcode{>=},
库提供了以下内容:

\indexlibrary{\idxcode{operator"!=}}%
\begin{itemdecl}
template<class T> bool operator!=(const T& x, const T& y);
\end{itemdecl}

\begin{itemdescr}
\pnum
\expects
\tcode{T} 满足 \oldconcept{EqualityComparable} 要求 (\tref{cpp17.equalitycomparable})。

\pnum
\returns
\tcode{!(x == y)}.
\end{itemdescr}

\indexlibraryglobal{operator>}%
\begin{itemdecl}
template<class T> bool operator>(const T& x, const T& y);
\end{itemdecl}

\begin{itemdescr}
\pnum
\expects
\tcode{T} 满足 \oldconcept{LessThanComparable} 要求 (\tref{cpp17.lessthancomparable})。

\pnum
\returns
\tcode{y < x}.
\end{itemdescr}

\indexlibrary{\idxcode{operator<=}}%
\begin{itemdecl}
template<class T> bool operator<=(const T& x, const T& y);
\end{itemdecl}

\begin{itemdescr}
\pnum
\expects
\tcode{T} 满足 \oldconcept{LessThanComparable} 要求 (\tref{cpp17.lessthancomparable})。

\pnum
\returns
\tcode{!(y < x)}.
\end{itemdescr}

\indexlibrary{\idxcode{operator>=}}%
\begin{itemdecl}
template<class T> bool operator>=(const T& x, const T& y);
\end{itemdecl}

\begin{itemdescr}
\pnum
\expects
\tcode{T} 满足 \oldconcept{LessThanComparable} 要求 (\tref{cpp17.lessthancomparable})。

\pnum
\returns
\tcode{!(x < y)}.
\end{itemdescr}

\rSec1[depr.tuple]{元组}

\pnum
头文件 \libheaderref{tuple} 具有以下附加内容:

\begin{codeblock}
namespace std {
  template<class T> struct tuple_size<volatile T>;
  template<class T> struct tuple_size<const volatile T>;

  template<size_t I, class T> struct tuple_element<I, volatile T>;
  template<size_t I, class T> struct tuple_element<I, const volatile T>;
}
\end{codeblock}

\begin{itemdecl}
template<class T> struct tuple_size<volatile T>;
template<class T> struct tuple_size<const volatile T>;
\end{itemdecl}

\begin{itemdescr}
\pnum
令 \tcode{TS} 表示 cv 非限定类型 \tcode{T} 的 \tcode{tuple_size<T>}。
如果表达式 \tcode{TS::value} 在
被视为未求值操作数\iref{term.unevaluated.operand} 时是良构的,
则每个模板的特化都满足
具有 \tcode{integral_constant<size_t, TS::value>} 的基本特征的 \oldconcept{TransformationTrait} 要求。
否则,它们没有成员 \tcode{value}。

\pnum
访问检查的执行方式
与 \tcode{TS} 和 \tcode{T} 无关的上下文相同。
仅考虑表达式的直接上下文的有效性。

\pnum
除了可以通过包含 \libheaderref{tuple} 头文件来使用外,
这两个模板在包含
\libheaderref{array}、
\libheaderref{ranges} 或
\libheaderref{utility}
头文件中的任何一个时也可用。
\end{itemdescr}

\begin{itemdecl}
template<size_t I, class T> struct tuple_element<I, volatile T>;
template<size_t I, class T> struct tuple_element<I, const volatile T>;
\end{itemdecl}

\begin{itemdescr}
\pnum
令 \tcode{TE} 表示 cv 非限定类型 \tcode{T} 的 \tcode{tuple_element_t<I, T>}。
然后,每个模板的特化都满足
\oldconcept{TransformationTrait} 要求,
其中成员 typedef \tcode{type} 命名以下类型:
\begin{itemize}
\item 对于第一个特化,\tcode{add_volatile_t<TE>},以及
\item 对于第二个特化,\tcode{add_cv_t<TE>}。
\end{itemize}

\pnum
除了可以通过包含 \libheaderref{tuple} 头文件来使用外,
这两个模板在包含
\libheaderref{array}、
\libheaderref{ranges} 或
\libheaderref{utility}
头文件中的任何一个时也可用。
\end{itemdescr}

\rSec1[depr.variant]{变体}

\pnum
头文件 \libheaderref{variant} 具有以下附加内容:

\begin{codeblock}
namespace std {
  template<class T> struct variant_size<volatile T>;
  template<class T> struct variant_size<const volatile T>;

  template<size_t I, class T> struct variant_alternative<I, volatile T>;
  template<size_t I, class T> struct variant_alternative<I, const volatile T>;
}
\end{codeblock}

\begin{itemdecl}
template<class T> struct variant_size<volatile T>;
template<class T> struct variant_size<const volatile T>;
\end{itemdecl}

\begin{itemdescr}
\pnum
令 \tcode{VS} 表示 cv 非限定类型 \tcode{T}
的 \tcode{variant_size<T>}。
然后,每个模板的特化都满足
具有 \tcode{integral_constant<size_t, VS::value>} 的基本特征的 \oldconcept{UnaryTypeTrait} 要求。
\end{itemdescr}

\begin{itemdecl}
template<size_t I, class T> struct variant_alternative<I, volatile T>;
template<size_t I, class T> struct variant_alternative<I, const volatile T>;
\end{itemdecl}

\begin{itemdescr}
\pnum
令 \tcode{VA} 表示 cv 非限定类型 \tcode{T}
的 \tcode{variant_alternative<I, T>}。
然后,每个模板的特化都满足
\oldconcept{TransformationTrait} 要求,
其中成员 typedef \tcode{type} 命名以下类型:
\begin{itemize}
\item 对于第一个特化,\tcode{add_volatile_t<VA::type>},以及
\item 对于第二个特化,\tcode{add_cv_t<VA::type>}。
\end{itemize}
\end{itemdescr}

\rSec1[depr.iterator]{不推荐使用的 \tcode{iterator} 类模板}

\pnum
头文件 \libheaderrefx{iterator}{iterator.synopsis} 具有以下附加内容:

\indexlibraryglobal{iterator}%
\begin{codeblock}
namespace std {
  template<class Category, class T, class Distance = ptrdiff_t,
           class Pointer = T*, class Reference = T&>
  struct iterator {
    using iterator_category = Category;
    using value_type        = T;
    using difference_type   = Distance;
    using pointer           = Pointer;
    using reference         = Reference;
  };
}
\end{codeblock}

\pnum
\tcode{iterator}
模板可以用作基类,以简化新迭代器所需类型的定义。

\pnum
\begin{note}
如果新迭代器类型是类模板,则这些别名
将无法从迭代器类的模板定义中看到,而只能
从该类的调用者中看到。
\end{note}

\pnum
\begin{example}
如果 \Cpp{} 程序想要为包含 \tcode{double} 的某个数据结构定义双向迭代器,并且该迭代器适用于实现的大内存模型,则可以使用以下方式实现:

\begin{codeblock}
class MyIterator :
  public iterator<bidirectional_iterator_tag, double, long, T*, T&> {
  // 实现 \tcode{++} 等的代码。
};
\end{codeblock}
\end{example}

\rSec1[depr.move.iter.elem]{不推荐使用的 \tcode{move_iterator} 访问}

\pnum
除了 \ref{move.iter.elem} 中指定的成员外,还声明了以下成员:

\begin{codeblock}
namespace std {
  template<class Iterator>
  class move_iterator {
  public:
    constexpr pointer operator->() const;
  };
}
\end{codeblock}

\indexlibrarymember{operator->}{move_iterator}%
\begin{itemdecl}
constexpr pointer operator->() const;
\end{itemdecl}

\begin{itemdescr}
\pnum
\returns
\tcode{current}.
\end{itemdescr}

\rSec1[depr.locale.category]{不推荐使用的 locale 类别方面}

\pnum
\tcode{ctype} locale 类别包括以下方面,
就像它们在 \ref{locale.category} 的 \tref{locale.category.facets} 中指定的一样。

\begin{codeblock}
codecvt<char16_t, char, mbstate_t>
codecvt<char32_t, char, mbstate_t>
codecvt<char16_t, char8_t, mbstate_t>
codecvt<char32_t, char8_t, mbstate_t>
\end{codeblock}

\pnum
\tcode{ctype} locale 类别包括以下方面,
就像它们在 \ref{locale.category} 的 \tref{locale.spec} 中指定的一样。

\begin{codeblock}
codecvt_byname<char16_t, char, mbstate_t>
codecvt_byname<char32_t, char, mbstate_t>
codecvt_byname<char16_t, char8_t, mbstate_t>
codecvt_byname<char32_t, char8_t, mbstate_t>
\end{codeblock}

\pnum
除了 \ref{locale.codecvt} 中指定的类模板特化外,还需要
以下类模板特化。
\indextext{UTF-8}%
\indextext{UTF-16}%
特化 \tcode{codecvt<char16_t, char, mbstate_t>} 和
\tcode{codecvt<char16_t, char8_t, mbstate_t>}
在 UTF-16 和 UTF-8 编码形式之间进行转换,并且
\indextext{UTF-32}%
特化 \tcode{codecvt<char32_t, char, mbstate_t>} 和
\tcode{codecvt<char32_t, char8_t, mbstate_t>}
在 UTF-32 和 UTF-8 编码形式之间进行转换。

\rSec1[depr.format]{不推荐使用的格式化}

\rSec2[depr.format.syn]{头文件 \tcode{<format>} 概要}

\pnum
头文件 \libheaderref{format} 具有以下附加内容:

\begin{codeblock}
namespace std {
  template<class Visitor, class Context>
    decltype(auto) visit_format_arg(Visitor&& vis, basic_format_arg<Context> arg);
}
\end{codeblock}

\rSec2[depr.format.arg]{格式化参数}

\indexlibraryglobal{visit_format_arg}%
\begin{itemdecl}
template<class Visitor, class Context>
  decltype(auto) visit_format_arg(Visitor&& vis, basic_format_arg<Context> arg);
\end{itemdecl}

\begin{itemdescr}
\pnum
\effects
等价于:\tcode{return visit(std::forward<Visitor>(vis), arg.value);}
\end{itemdescr}

\rSec1[depr.fs.path.factory]{不推荐使用的 filesystem 路径工厂函数}

\pnum
头文件 \libheaderrefx{filesystem}{fs.filesystem.syn} 具有以下附加内容:

\indexlibraryglobal{u8path}%
\begin{itemdecl}
template<class Source>
  path u8path(const Source& source);
template<class InputIterator>
  path u8path(InputIterator first, InputIterator last);
\end{itemdecl}

\begin{itemdescr}
\pnum
\mandates
\tcode{Source} 和 \tcode{InputIterator} 的值类型为
\tcode{char} 或 \keyword{char8_t}。

\pnum
\expects
\tcode{source} 和 \range{first}{last} 序列是 UTF-8 编码的。
\tcode{Source} 满足 \ref{fs.path.req} 中指定的要求。

\pnum
\returns
\begin{itemize}
\item 如果 \tcode{path::value_type} 是 \tcode{char} 并且当前的本地
      窄编码\iref{fs.path.type.cvt} 是 UTF-8,
      返回 \tcode{path(source)} 或 \tcode{path(first, last)};
      否则,
\item 如果 \tcode{path::value_type} 是 \keyword{wchar_t} 并且
      本地宽编码是 UTF-16,或者
      如果 \tcode{path::value_type} 是 \keyword{char16_t} 或 \keyword{char32_t},
      将 \tcode{source} 或 \range{first}{last}
      转换为类型为 \tcode{path::string_type} 的临时变量 \tcode{tmp},
      并返回 \tcode{path(tmp)};
      否则,
\item 将 \tcode{source} 或 \range{first}{last}
      转换为类型为 \tcode{u32string} 的临时变量 \tcode{tmp},
      并返回 \tcode{path(tmp)}。
\end{itemize}

\pnum
\remarks
参数格式转换\iref{fs.path.fmt.cvt} 适用于
  这些函数的参数。如何执行 Unicode 编码转换是
  未指定的。

\pnum
\begin{example}
要从以 UTF-8 编码的数据库中读取字符串,并使用
    文件名的本地编码创建目录:
\begin{codeblock}
namespace fs = std::filesystem;
std::string utf8_string = read_utf8_data();
fs::create_directory(fs::u8path(utf8_string));
\end{codeblock}

对于将本地窄编码设置为 UTF-8 的基于 POSIX 的操作系统,
    不会发生编码或类型转换。

对于未将本地窄编码设置为 UTF-8 的基于 POSIX 的操作系统,
    会发生到 UTF-32 的转换,然后转换为
    当前的本地窄编码。某些 Unicode 字符可能没有本地字符
    集表示。

对于基于 Windows 的操作系统,会发生从 UTF-8 到
    UTF-16 的转换。
\end{example}
\begin{note}
上面的示例代表了
    \tcode{filesystem::u8path} 的历史用法。
为了指示 UTF-8 编码,
    首选将 \tcode{std::u8string} 传递给 \tcode{path} 的构造函数,
因为它与 \tcode{path} 对其他编码的处理方式一致。
\end{note}
\end{itemdescr}

\rSec1[depr.atomics]{不推荐使用的原子操作}

\rSec2[depr.atomics.general]{概述}

\pnum
头文件 \libheaderrefx{atomic}{atomics.syn} 具有以下附加内容。

\begin{codeblock}
namespace std {
  template<class T>
    void atomic_init(volatile atomic<T>*, typename atomic<T>::value_type) noexcept;
  template<class T>
    void atomic_init(atomic<T>*, typename atomic<T>::value_type) noexcept;

  #define @\libmacro{ATOMIC_VAR_INIT}@(value) @\seebelow@
}
\end{codeblock}

\rSec2[depr.atomics.volatile]{Volatile 访问}

\pnum
如果 \tcode{atomic}\iref{atomics.types.generic} 特化具有以下重载之一,
则即使 \tcode{atomic<T>::is_always_lock_free} 为 \tcode{false},该重载也会参与重载决议:
\begin{codeblock}
void store(T desired, memory_order order = memory_order::seq_cst) volatile noexcept;
T operator=(T desired) volatile noexcept;
T load(memory_order order = memory_order::seq_cst) const volatile noexcept;
operator T() const volatile noexcept;
T exchange(T desired, memory_order order = memory_order::seq_cst) volatile noexcept;
bool compare_exchange_weak(T& expected, T desired,
                           memory_order success, memory_order failure) volatile noexcept;
bool compare_exchange_strong(T& expected, T desired,
                             memory_order success, memory_order failure) volatile noexcept;
bool compare_exchange_weak(T& expected, T desired,
                           memory_order order = memory_order::seq_cst) volatile noexcept;
bool compare_exchange_strong(T& expected, T desired,
                             memory_order order = memory_order::seq_cst) volatile noexcept;
T fetch_@\placeholdernc{key}@(T operand, memory_order order = memory_order::seq_cst) volatile noexcept;
T operator @\placeholdernc{op}@=(T operand) volatile noexcept;
T* fetch_@\placeholdernc{key}@(ptrdiff_t operand, memory_order order = memory_order::seq_cst) volatile noexcept;
\end{codeblock}

\rSec2[depr.atomics.nonmembers]{非成员函数}

\indexlibraryglobal{atomic_init}%
\begin{itemdecl}
template<class T>
  void atomic_init(volatile atomic<T>* object, typename atomic<T>::value_type desired) noexcept;
template<class T>
  void atomic_init(atomic<T>* object, typename atomic<T>::value_type desired) noexcept;
\end{itemdecl}

\begin{itemdescr}
\pnum
\effects
等价于:\tcode{atomic_store_explicit(object, desired, memory_order::relaxed);}
\end{itemdescr}

\rSec2[depr.atomics.types.operations]{原子类型上的操作}

\indexlibraryglobal{ATOMIC_VAR_INIT}%
\begin{itemdecl}
#define @\libmacro{ATOMIC_VAR_INIT}@(value) @\seebelow@
\end{itemdecl}

\begin{itemdescr}
\pnum
该宏展开为一个记号序列,该序列适用于
静态存储期的原子变量的常量初始化,该原子变量的类型
与 \tcode{value} 初始化兼容。
\begin{note}
此操作可能需要初始化锁。
\end{note}
对正在初始化的变量的并发访问,
即使通过原子操作,
也构成数据竞争。
\begin{example}
\begin{codeblock}
atomic<int> v = ATOMIC_VAR_INIT(5);
\end{codeblock}
\end{example}
\end{itemdescr}
