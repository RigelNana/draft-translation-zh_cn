%!TEX root = std.tex
\rSec0[stmt.stmt]{语句}%
\indextext{statement|(}

\gramSec[gram.stmt]{语句}

\indextext{block (statement)|see{statement, compound}}

\rSec1[stmt.pre]{前言}

\pnum
除非另有说明,语句按顺序执行\iref{intro.execution}。

\begin{bnf}
\nontermdef{statement}\br
    labeled-statement\br
    \opt{attribute-specifier-seq} expression-statement\br
    \opt{attribute-specifier-seq} compound-statement\br
    \opt{attribute-specifier-seq} selection-statement\br
    \opt{attribute-specifier-seq} iteration-statement\br
    \opt{attribute-specifier-seq} jump-statement\br
    declaration-statement\br
    \opt{attribute-specifier-seq} try-block
\end{bnf}

\begin{bnf}
\nontermdef{init-statement}\br
    expression-statement\br
    simple-declaration\br
    alias-declaration
\end{bnf}

\begin{bnf}
\nontermdef{condition}\br
    expression\br
    \opt{attribute-specifier-seq} decl-specifier-seq declarator brace-or-equal-initializer\br
    structured-binding-declaration initializer
\end{bnf}

可选的 \grammarterm{attribute-specifier-seq} 属于相应的语句。

\pnum
\grammarterm{statement} 的 \defn{子语句} 是以下之一:
\begin{itemize}
\item
  对于 \grammarterm{labeled-statement},是其 \grammarterm{statement},
\item
  对于 \grammarterm{compound-statement},是其 \grammarterm{statement-seq} 的任何 \grammarterm{statement},
\item
  对于 \grammarterm{selection-statement},是其任何 \grammarterm{statement} 或 \grammarterm{compound-statement}(但不是其 \grammarterm{init-statement}),或者
\item
  对于 \grammarterm{iteration-statement},是其 \grammarterm{statement}(但不是 \grammarterm{init-statement})。
\end{itemize}
\begin{note}
\grammarterm{lambda-expression} 的 \grammarterm{compound-statement} 不是 \grammarterm{lambda-expression} 在词法上出现的 \grammarterm{statement}(如果有)的子语句。
\end{note}

\pnum
\indextext{statement!enclosing}%
如果满足以下条件,则 \grammarterm{statement} \tcode{S1} \defnx{包围}{enclosing statement} \grammarterm{statement} \tcode{S2}:
\begin{itemize}
\item
  \tcode{S2} 是 \tcode{S1} 的子语句,
\item
  \tcode{S1} 是 \grammarterm{selection-statement} 或 \grammarterm{iteration-statement},并且 \tcode{S2} 是 \tcode{S1} 的 \grammarterm{init-statement},
\item
  \tcode{S1} 是 \grammarterm{try-block} 并且 \tcode{S2} 是其 \grammarterm{compound-statement} 或其 \grammarterm{handler} 的任何 \grammarterm{compound-statement},或者
\item
  \tcode{S1} 包围语句 \tcode{S3} 并且 \tcode{S3} 包围 \tcode{S2}。
\end{itemize}
\indextext{statement!enclosed by}%
如果 \tcode{S2} 包围 \tcode{S1},则语句 \tcode{S1} \defnx{被}{enclosed by statement} 语句 \tcode{S2} 包围。

\pnum
\indextext{\idxgram{condition}{s}!rules for}%
\grammarterm{condition} 的规则既适用于 \grammarterm{selection-statement}\iref{stmt.select},也适用于 \keyword{for} 和 \keyword{while} 语句\iref{stmt.iter}。如果 \grammarterm{structured-binding-declaration} 出现在 \grammarterm{condition} 中,则 \grammarterm{condition} 是结构化绑定声明\iref{dcl.pre}。既不是 \grammarterm{expression} 也不是结构化绑定声明的 \grammarterm{condition} 是声明\iref{dcl.dcl}。\grammarterm{declarator} 不应指定函数或数组。\grammarterm{decl-specifier-seq} 不应定义类或枚举。如果 \keyword{auto} \grammarterm{type-specifier} 出现在 \grammarterm{decl-specifier-seq} 中,则被声明标识符的类型从初始化器中推导出来,如 \ref{dcl.spec.auto} 中所述。

\pnum
既不是 \grammarterm{expression} 也不是结构化绑定声明的 \grammarterm{condition} 的\defnadj{决策}{variable} 变量是被声明的变量。\grammarterm{condition}(作为结构化绑定声明)的决策变量在 \ref{dcl.struct.bind} 中指定。

\pnum
在 \keyword{switch} 语句之外的语句中,不是 \grammarterm{expression} 的 \grammarterm{condition} 的值是上下文转换为 \tcode{bool}\iref{conv} 的决策变量的值。如果该转换是非良构的,则程序是非良构的。对于 \keyword{switch} 之外的语句,作为 \grammarterm{expression} 的 \grammarterm{condition} 的值是表达式的值,上下文转换为 \tcode{bool};如果该转换是非良构的,则程序是非良构的。在用法明确的情况下,条件的值将被简称为“条件”。

\pnum
如果 \grammarterm{condition} 可以在语法上解析为表达式或声明,则将其解释为后者。

\pnum
在 \grammarterm{condition} 的 \grammarterm{decl-specifier-seq} 中,包括 \grammarterm{condition} 的任何 \grammarterm{structured-binding-declaration} 的 \grammarterm{decl-specifier-seq} 中,每个 \grammarterm{decl-specifier} 都应是 \grammarterm{type-specifier} 或 \keyword{constexpr}。

\rSec1[stmt.label]{标签}%
\indextext{statement!labeled}

\pnum
\indextext{statement!labeled}%
\indextext{\idxcode{:}!label specifier}%
标签可以添加到语句中,也可以在 \grammarterm{compound-statement} 中的任何位置使用。

\begin{bnf}
\nontermdef{label}\br
    \opt{attribute-specifier-seq} identifier \terminal{:}\br
    \opt{attribute-specifier-seq} \keyword{case} constant-expression \terminal{:}\br
    \opt{attribute-specifier-seq} \keyword{default} \terminal{:}
\end{bnf}

\begin{bnf}
\nontermdef{labeled-statement}\br
    label statement
\end{bnf}

可选的 \grammarterm{attribute-specifier-seq} 属于标签。
\indextext{statement!\idxcode{goto}}%
带有 \grammarterm{identifier} 的标签的唯一用途是作为 \tcode{goto} 的目标。
\indextext{label!scope of}%
函数中不得有两个标签具有相同的 \grammarterm{identifier}。标签可以在其引入之前在 \tcode{goto} 语句中使用。

\pnum
\indextext{\idxgram{labeled-statement}}%
\indextext{label!\idxcode{case}}%
\indextext{label!\idxcode{default}}%
其 \grammarterm{label} 是 \keyword{case} 或 \keyword{default} 标签的 \grammarterm{labeled-statement} 应被 \keyword{switch} 语句\iref{stmt.switch} 包围\iref{stmt.pre}。

\pnum
\defnadj{控制流受限}{statement} 语句是语句 \tcode{S},其中:
\begin{itemize}
\item
  出现在 \tcode{S} 内的 \keyword{case} 或 \keyword{default} 标签应与 \tcode{S} 内的 \keyword{switch} 语句\iref{stmt.switch} 相关联,并且
\item
  在 \tcode{S} 中声明的标签只能被 \tcode{S} 中的 \grammarterm{statement}\iref{stmt.goto} 引用。
\end{itemize}


\rSec1[stmt.expr]{表达式语句}%
\indextext{statement!expression}

\pnum
表达式语句的形式为
\begin{bnf}
\nontermdef{expression-statement}\br
    \opt{expression} \terminal{;}
\end{bnf}

该表达式是一个被丢弃值的表达式\iref{expr.context}。表达式语句中的所有
\indextext{side effects}%
副作用在执行下一条语句之前完成。
\indextext{statement!empty}%
缺少 \grammarterm{expression} 的表达式语句称为 \defnadj{空}{statement} 语句。
\begin{note}
大多数语句都是表达式语句 --- 通常是赋值或函数调用。空语句对于向诸如 \keyword{while} 语句\iref{stmt.while} 之类的迭代语句提供空主体很有用。
\end{note}

\rSec1[stmt.block]{复合语句或块}%
\indextext{\idxcode{\{\}}!block statement}%

\pnum
\defnadj{复合}{statement} 语句(也称为块)将一系列语句组合成单个语句。

\begin{bnf}
\nontermdef{compound-statement}\br
    \terminal{\{} \opt{statement-seq} \opt{label-seq} \terminal{\}}
\end{bnf}

\begin{bnf}
\nontermdef{statement-seq}\br
    statement \opt{statement-seq}
\end{bnf}

\begin{bnf}
\nontermdef{label-seq}\br
    label \opt{label-seq}
\end{bnf}

\grammarterm{compound-statement} 末尾的标签被视为后跟一个空语句。

\pnum
\begin{note}
复合语句定义了一个块作用域\iref{basic.scope}。声明是一个 \grammarterm{statement}\iref{stmt.dcl}。
\end{note}

\rSec1[stmt.select]{选择语句}%

\rSec2[stmt.select.general]{概述}%
\indextext{statement!selection|(}

\pnum
选择语句选择多个控制流之一。

\indextext{statement!\idxcode{if}}%
\indextext{statement!\idxcode{switch}}%
%
\begin{bnf}
\nontermdef{selection-statement}\br
    \keyword{if} \opt{\keyword{constexpr}} \terminal{(} \opt{init-statement} condition \terminal{)} statement\br
    \keyword{if} \opt{\keyword{constexpr}} \terminal{(} \opt{init-statement} condition \terminal{)} statement \keyword{else} statement\br
    \keyword{if} \opt{\terminal{!}} \keyword{consteval} compound-statement\br
    \keyword{if} \opt{\terminal{!}} \keyword{consteval} compound-statement \keyword{else} statement\br
    \keyword{switch} \terminal{(} \opt{init-statement} condition \terminal{)} statement
\end{bnf}

有关条件中的可选 \grammarterm{attribute-specifier-seq},请参见 \ref{dcl.meaning}。
\begin{note}
\grammarterm{init-statement} 以分号结尾。
\end{note}

\pnum
\indextext{scope!\idxgram{selection-statement}}%
\begin{note}
每个 \grammarterm{selection-statement} 和 \grammarterm{selection-statement} 的每个子语句都有一个块作用域\iref{basic.scope.block}。
\end{note}

\rSec2[stmt.if]{ \keyword{if} 语句}%
\indextext{statement!\idxcode{if}}

\pnum
如果 condition\iref{stmt.pre} 产生 \tcode{true},则执行第一个子语句。如果选择语句的 \keyword{else} 部分存在并且 condition 产生 \tcode{false},则执行第二个子语句。如果通过标签到达第一个子语句,则不评估 condition,也不执行第二个子语句。在第二种形式的 \keyword{if} 语句(包含 \keyword{else} 的语句)中,如果第一个子语句也是一个 \keyword{if} 语句,则该内部 \tcode{if} 语句应包含一个 \keyword{else} 部分。
\begin{footnote}
换句话说,\keyword{else} 与最近的未配对 \keyword{else} 的 \keyword{if} 相关联。
\end{footnote}

\pnum
如果 \keyword{if} 语句的形式为 \tcode{if constexpr},则 condition 的值被上下文转换为 \keyword{bool},并且转换后的表达式应为常量表达式\iref{expr.const};这种形式称为 \defn{constexpr if} 语句。如果转换后的 condition 的值为 \tcode{false},则第一个子语句是 \defn{discarded statement},否则第二个子语句(如果存在)是 discarded statement。在封闭的模板化实体\iref{temp.pre} 的实例化期间,如果 condition 在实例化后不是值依赖的,则 discarded statement(如果有)不会被实例化。constexpr if 语句的每个子语句都是控制流受限语句\iref{stmt.label}。
\begin{example}
\begin{codeblock}
if constexpr (sizeof(int[2])) {}        // OK, 允许窄化
\end{codeblock}
\end{example}
\begin{note}
discarded statement 中的 odr-use\iref{term.odr.use} 不需要定义实体。
\end{note}
\begin{example}
\begin{codeblock}
template<typename T, typename ... Rest> void g(T&& p, Rest&& ...rs) {
  // ... 处理 \tcode{p}

  if constexpr (sizeof...(rs) > 0)
    g(rs...);       // 从不使用空参数列表实例化
}

extern int x;       // 不需要 \tcode{x} 的定义

int f() {
  if constexpr (true)
    return 0;
  else if (x)
    return x;
  else
    return -x;
}
\end{codeblock}
\end{example}

\pnum
形式为
\begin{ncsimplebnf}
\keyword{if} \opt{\keyword{constexpr}} \terminal{(} init-statement condition \terminal{)} statement
\end{ncsimplebnf}
的 \keyword{if} 语句等价于
\begin{ncsimplebnf}
\terminal{\{}\br
\bnfindent init-statement\br
\bnfindent \keyword{if} \opt{\keyword{constexpr}} \terminal{(} condition \terminal{)} statement\br
\terminal{\}}
\end{ncsimplebnf}
形式为
\begin{ncsimplebnf}
\keyword{if} \opt{\keyword{constexpr}} \terminal{(} init-statement condition \terminal{)} statement \keyword{else} statement
\end{ncsimplebnf}
的 \keyword{if} 语句等价于
\begin{ncsimplebnf}
\terminal{\{}\br
\bnfindent init-statement\br
\bnfindent \keyword{if} \opt{\keyword{constexpr}} \terminal{(} condition \terminal{)} statement \keyword{else} statement\br
\terminal{\}}
\end{ncsimplebnf}
除了 \grammarterm{init-statement} 与 \grammarterm{condition} 在同一作用域中。

\pnum
形式为 \tcode{\keyword{if} \keyword{consteval}} 的 \keyword{if} 语句称为 \defnadj{consteval if}{statement} 语句。consteval if 语句中的 \grammarterm{statement}(如果有)应为 \grammarterm{compound-statement}。
\begin{example}
\begin{codeblock}
constexpr void f(bool b) {
  if (true)
    if consteval { }
    else ;              // 错误:不是 \grammarterm{compound-statement};\keyword{else} 未与外部 \keyword{if} 关联
}
\end{codeblock}
\end{example}

\pnum
如果 consteval if 语句在明显常量求值的上下文\iref{expr.const}中求值,则执行第一个子语句。
\begin{note}
第一个子语句是立即函数上下文。
\end{note}
否则,如果选择语句的 \keyword{else} 部分存在,则执行第二个子语句。consteval if 语句的每个子语句都是控制流受限语句\iref{stmt.label}。

\pnum
形式为
\begin{ncsimplebnf}
\keyword{if} \terminal{!} \keyword{consteval} compound-statement
\end{ncsimplebnf}
的 \keyword{if} 语句本身不是 consteval if 语句,但等价于 consteval if 语句
\begin{ncsimplebnf}
\keyword{if} \keyword{consteval} \terminal{\{} \terminal{\}} \keyword{else} compound-statement
\end{ncsimplebnf}
形式为
\begin{ncsimplebnf}
\keyword{if} \terminal{!} \keyword{consteval} compound-statement$_1$ \keyword{else} statement$_2$
\end{ncsimplebnf}
的 \keyword{if} 语句本身不是 consteval if 语句,但等价于 consteval if 语句
\begin{ncsimplebnf}
\keyword{if} \keyword{consteval} statement$_2$ \keyword{else} compound-statement$_1$
\end{ncsimplebnf}

\rSec2[stmt.switch]{ \keyword{switch} 语句}%
\indextext{statement!\idxcode{switch}}

\pnum
\keyword{switch} 语句根据 condition 的值使控制转移到多个语句之一。

\pnum
如果 \grammarterm{condition} 是 \grammarterm{expression},则 condition 的值是 \grammarterm{expression} 的值;否则,它是决策变量的值。condition 的值应为整数类型、枚举类型或类类型。如果是类类型,则 condition 上下文隐式转换\iref{conv}为整数或枚举类型。如果(可能转换后的)类型受整数提升\iref{conv.prom}的约束,则 condition 将转换为提升后的类型。 \keyword{switch} 语句中的任何语句都可以用一个或多个 case 标签标记,如下所示:
\begin{ncbnf}
\indextext{label!\idxcode{case}}%
\keyword{case} constant-expression \terminal{:}
\end{ncbnf}
其中 \grammarterm{constant-expression} 应为 switch condition 调整后类型的转换常量表达式\iref{expr.const}。同一 switch 中的两个 case 常量在转换后不得具有相同的值。


\pnum
\indextext{label!\idxcode{default}}%
在一个 \keyword{switch} 语句中,至多应有一个形式为
\begin{codeblock}
default :
\end{codeblock}
的标签。

\pnum
Switch 语句可以嵌套;\keyword{case} 或 \keyword{default} 标签与包含它的最小 switch 相关联。

\pnum
当执行 \keyword{switch} 语句时,将评估其 condition。
\indextext{label!\idxcode{case}}%
如果其中一个 case 常量的值与 condition 的值相同,则控制权将传递到匹配的 case 标签后面的语句。如果没有 case 常量与 condition 匹配,并且如果存在
\indextext{label!\idxcode{default}}%
\keyword{default} 标签,则控制权传递到由 default 标签标记的语句。如果没有 case 匹配并且没有 \keyword{default},则不执行 switch 中的任何语句。

\pnum
\keyword{case} 和 \keyword{default} 标签本身不会改变控制流,控制流会不受阻碍地跨越这些标签继续进行。要从 switch 中退出,请参见 \keyword{break},\ref{stmt.break}。
\begin{note}
通常,作为 switch 主体的子语句是复合的,并且 \keyword{case} 和 \keyword{default} 标签出现在包含在(复合)子语句中的顶层语句上,但这不是必需的。
\indextext{statement!declaration in \tcode{switch}}%
声明可以出现在 \keyword{switch} 语句的子语句中。
\end{note}

\pnum
形式为
\begin{ncsimplebnf}
\keyword{switch} \terminal{(} init-statement condition \terminal{)} statement
\end{ncsimplebnf}
的 \keyword{switch} 语句等价于
\begin{ncsimplebnf}
\terminal{\{}\br
\bnfindent init-statement\br
\bnfindent \keyword{switch} \terminal{(} condition \terminal{)} statement\br
\terminal{\}}
\end{ncsimplebnf}
除了 \grammarterm{init-statement} 与 \grammarterm{condition} 在同一作用域中。

\indextext{statement!selection|)}

\rSec1[stmt.iter]{迭代语句}%

\rSec2[stmt.iter.general]{概述}%
\indextext{statement!iteration|(}

\pnum
迭代语句指定循环。

\indextext{statement!\idxcode{while}}%
\indextext{statement!\idxcode{do}}%
\indextext{statement!\idxcode{for}}%
%
\begin{bnf}
\nontermdef{iteration-statement}\br
    \keyword{while} \terminal{(} condition \terminal{)} statement\br
    \keyword{do} statement \keyword{while} \terminal{(} expression \terminal{)} \terminal{;}\br
    \keyword{for} \terminal{(} init-statement \opt{condition} \terminal{;} \opt{expression} \terminal{)} statement\br
    \keyword{for} \terminal{(} \opt{init-statement} for-range-declaration \terminal{:} for-range-initializer \terminal{)} statement
\end{bnf}

\begin{bnf}
\nontermdef{for-range-declaration}\br
    \opt{attribute-specifier-seq} decl-specifier-seq declarator\br
    structured-binding-declaration
\end{bnf}

\begin{bnf}
\nontermdef{for-range-initializer}\br
    expr-or-braced-init-list
\end{bnf}

有关 \grammarterm{for-range-declaration} 中可选的 \grammarterm{attribute-specifier-seq},请参见 \ref{dcl.meaning}。
\begin{note}
\grammarterm{init-statement} 以分号结尾。
\end{note}

\pnum
\indextext{scope!\idxgram{iteration-statement}}%
\grammarterm{iteration-statement} 中的子语句隐式定义了一个块作用域\iref{basic.scope},每次循环都会进入和退出该作用域。如果 \grammarterm{iteration-statement} 中的子语句是单个语句而不是 \grammarterm{compound-statement},则等同于将其重写为包含原始语句的 \grammarterm{compound-statement}。
\begin{example}
\begin{codeblock}
while (--x >= 0)
  int i;
\end{codeblock}
可以等价地重写为
\begin{codeblock}
while (--x >= 0) {
  int i;
}
\end{codeblock}
因此,在 \keyword{while} 语句之后,\tcode{i} 不再在作用域内。
\end{example}

\pnum
\defnadj{平凡空}{iteration statement} 迭代语句是匹配以下形式之一的迭代语句:
\begin{itemize}
\item \tcode{while (} \grammarterm{expression} \tcode{) ;}
\item \tcode{while (} \grammarterm{expression} \tcode{) \{ \}}
\item \tcode{do ; while (} \grammarterm{expression} \tcode{) ;}
\item \tcode{do \{ \} while (} \grammarterm{expression} \tcode{) ;}
\item \tcode{for (} \grammarterm{init-statement} \opt{\grammarterm{expression}} \tcode{; ) ;}
\item \tcode{for (} \grammarterm{init-statement} \opt{\grammarterm{expression}} \tcode{; ) \{ \}}
\end{itemize}
平凡空迭代语句的\defnadj{控制}{expression} 表达式是 \tcode{while}、\tcode{do} 或 \tcode{for} 语句的 \grammarterm{expression}(如果 \tcode{for} 语句没有 \grammarterm{expression},则为 \tcode{true})。当解释为 \grammarterm{constant-expression}\iref{expr.const} 时,\defnadj{平凡无限}{loop} 循环是一个平凡空迭代语句,其转换后的控制表达式是一个常量表达式,并且计算结果为 \tcode{true}。平凡无限循环的 \grammarterm{statement} 被替换为对函数 \tcode{std::this_thread::yield}\iref{thread.thread.this} 的调用;在独立实现上是否发生此替换是实现定义的。
\begin{note}
在独立环境中,不保证并发向前推进;因此,此类系统需要显式协作。在没有其他意图的情况下,调用 yield 可以添加隐式协作。
\end{note}

\rSec2[stmt.while]{ \keyword{while} 语句}%
\indextext{statement!\idxcode{while}}

\pnum
在 \keyword{while} 语句中,重复执行子语句,直到 condition\iref{stmt.pre} 的值变为 \tcode{false}。测试发生在每次执行子语句之前。

\pnum
\indextext{statement!declaration in \tcode{while}}%
\keyword{while} 语句等价于
\begin{ncsimplebnf}
\exposid{label} \terminal{:}\br
\terminal{\{}\br
\bnfindent \keyword{if} \terminal{(} condition \terminal{)} \terminal{\{}\br
\bnfindent \bnfindent statement\br
\bnfindent \bnfindent \keyword{goto} \exposid{label} \terminal{;}\br
\bnfindent \terminal{\}}\br
\terminal{\}}
\end{ncsimplebnf}
\begin{note}
在 condition 中创建的变量在每次循环迭代时都会被销毁和创建。
\begin{example}
\begin{codeblock}
struct A {
  int val;
  A(int i) : val(i) { }
  ~A() { }
  operator bool() { return val != 0; }
};
int i = 1;
while (A a = i) {
  // ...
  i = 0;
}
\end{codeblock}
在 while 循环中,构造函数和析构函数分别被调用两次,一次用于成功的 condition,一次用于失败的 condition。
\end{example}
\end{note}

\rSec2[stmt.do]{ \keyword{do} 语句}%
\indextext{statement!\idxcode{do}}

\pnum
表达式被上下文转换为 \tcode{bool}\iref{conv};如果该转换是非良构的,则程序是非良构的。

\pnum
在 \keyword{do} 语句中,重复执行子语句,直到 expression 的值变为 \tcode{false}。测试发生在每次执行语句之后。

\rSec2[stmt.for]{ \tcode{for} 语句}%
\indextext{statement!\idxcode{for}}

\pnum
\keyword{for} 语句
\begin{ncsimplebnf}
\keyword{for} \terminal{(} init-statement \opt{condition} \terminal{;} \opt{expression} \terminal{)} statement
\end{ncsimplebnf}
等价于
\begin{ncsimplebnf}
\terminal{\{}\br
\bnfindent init-statement\br
\bnfindent \keyword{while} \terminal{(} condition \terminal{)} \terminal{\{}\br
\bnfindent\bnfindent statement\br
\bnfindent\bnfindent expression \terminal{;}\br
\bnfindent \terminal{\}}\br
\terminal{\}}
\end{ncsimplebnf}
除了 \grammarterm{init-statement} 与 \grammarterm{condition} 在同一作用域中,并且除了 \grammarterm{statement} 中的 \keyword{continue}(不包含在另一个迭代语句中)将在重新评估 \grammarterm{condition} 之前执行 \grammarterm{expression}。
\begin{note}
因此,第一个语句指定循环的初始化;condition\iref{stmt.pre} 指定一个测试,该测试在每次迭代之前按顺序进行,以便当 condition 变为 \tcode{false} 时退出循环;expression 通常指定在每次迭代之后按顺序进行的递增。
\end{note}

\pnum
\grammarterm{condition} 和 \grammarterm{expression} 中的一个或两个都可以省略。缺少的 \grammarterm{condition} 使隐含的 \keyword{while} 子句等价于 \tcode{while(true)}。

\rSec2[stmt.ranged]{基于范围的 \keyword{for} 语句}%
\indextext{statement!range based for@range based \tcode{for}}

\pnum
基于范围的 \keyword{for} 语句
\begin{ncsimplebnf}
\keyword{for} \terminal{(} \opt{init-statement} for-range-declaration \terminal{:} for-range-initializer \terminal{)} statement
\end{ncsimplebnf}
等价于
\begin{ncsimplebnf}
\terminal{\{}\br
\bnfindent \opt{init-statement}\br
\bnfindent \keyword{auto} \terminal{\&\&}\exposid{range} \terminal{=} for-range-initializer \terminal{;}\br
\bnfindent \keyword{auto} \exposid{begin} \terminal{=} \exposid{begin-expr} \terminal{;}\br
\bnfindent \keyword{auto} \exposid{end} \terminal{=} \exposid{end-expr} \terminal{;}\br
\bnfindent \keyword{for} \terminal{(} \terminal{;} \exposid{begin} \terminal{!=} \exposid{end}\terminal{;} \terminal{++}\exposid{begin} \terminal{)} \terminal{\{}\br
\bnfindent\bnfindent for-range-declaration \terminal{=} \terminal{*} \exposid{begin} \terminal{;}\br
\bnfindent\bnfindent statement\br
\bnfindent \terminal{\}}\br
\terminal{\}}
\end{ncsimplebnf}
其中
\begin{itemize}
\item
如果 \grammarterm{for-range-initializer} 是 \grammarterm{expression},则将其视为被括号包围(因此逗号运算符不能被重新解释为分隔两个 \grammarterm{init-declarator});

\item \exposid{range}、\exposid{begin} 和 \exposid{end} 是仅为说明而定义的变量;并且

\item
\exposid{begin-expr} 和 \exposid{end-expr} 确定如下:

\begin{itemize}
\item 如果 \exposid{range} 的类型是对数组类型 \tcode{R} 的引用,则 \exposid{begin-expr} 和 \exposid{end-expr} 分别是 \exposid{range} 和 \exposid{range} \tcode{+} \tcode{N},其中 \tcode{N} 是数组边界。如果 \tcode{R} 是未知边界的数组或不完整类型的数组,则程序是非良构的;

\item 如果 \exposid{range} 的类型是对类类型 \tcode{C} 的引用,并且在 \tcode{C}\iref{class.member.lookup} 的作用域中搜索名称 \tcode{begin} 和 \tcode{end} 都至少找到一个声明,则 \exposid{begin-expr} 和 \exposid{end-expr} 分别是 \tcode{\exposid{range}.begin()} 和 \tcode{\exposid{range}.end()};

\item 否则,\exposid{begin-expr} 和 \exposid{end-expr} 分别是 \tcode{begin(\exposid{range})} 和 \tcode{end(\exposid{range})},其中 \tcode{begin} 和 \tcode{end} 进行参数依赖查找\iref{basic.lookup.argdep}。
\begin{note}
不执行普通的非限定查找\iref{basic.lookup.unqual}。
\end{note}
\end{itemize}
\end{itemize}
\begin{example}
\begin{codeblock}
int array[5] = { 1, 2, 3, 4, 5 };
for (int& x : array)
  x *= 2;
\end{codeblock}
\end{example}
\begin{note}
\grammarterm{for-range-initializer} 中某些临时对象的生命周期被延长以覆盖整个循环\iref{class.temporary}。
\end{note}
\begin{example}
\begin{codeblock}
using T = std::list<int>;
const T& f1(const T& t) { return t; }
const T& f2(T t)        { return t; }
T g();

void foo() {
  for (auto e : f1(g())) {}     // OK, \tcode{g()} 返回值的生命周期被延长
  for (auto e : f2(g())) {}     // 未定义行为
}
\end{codeblock}
\end{example}

\pnum
在 \grammarterm{for-range-declaration} 的 \grammarterm{decl-specifier-seq} 中,每个 \grammarterm{decl-specifier} 都应为 \grammarterm{type-specifier} 或 \keyword{constexpr}。\grammarterm{decl-specifier-seq} 不应定义类或枚举。%
\indextext{statement!iteration|)}

\rSec1[stmt.jump]{跳转语句}%

\rSec2[stmt.jump.general]{概述}%
\indextext{statement!jump}

\pnum
跳转语句无条件地转移控制。
\indextext{statement!jump}%

\indextext{statement!\idxcode{break}}%
\indextext{statement!\idxcode{continue}}%
\indextext{return statement@\tcode{return} statement|see{\tcode{return}}}%
\indextext{\idxcode{return}}%
\indextext{statement!\idxcode{goto}}%
%
\begin{bnf}
\nontermdef{jump-statement}\br
    \keyword{break} \terminal{;}\br
    \keyword{continue} \terminal{;}\br
    \keyword{return} \opt{expr-or-braced-init-list} \terminal{;}\br
    coroutine-return-statement\br
    \keyword{goto} identifier \terminal{;}
\end{bnf}

\pnum
\indextext{local variable!destruction of}%
\indextext{scope!destructor and exit from}%
\begin{note}
退出作用域时(无论以何种方式完成),在该作用域中构造的具有自动存储期\iref{basic.stc.auto}的对象将按其构造的相反顺序销毁\iref{stmt.dcl}。对于临时对象,请参见 \ref{class.temporary}。但是,程序可以在不销毁具有自动存储期的对象的情况下终止(例如,通过调用
\indextext{\idxcode{exit}}%
\indexlibraryglobal{exit}%
\tcode{std::exit()} 或
\indextext{\idxcode{abort}}%
\indexlibraryglobal{abort}%
\tcode{std::abort()}\iref{support.start.term})。
\end{note}
\begin{note}
协程的挂起\iref{expr.await}不被视为退出作用域。
\end{note}

\rSec2[stmt.break]{ \keyword{break} 语句}%
\indextext{statement!\idxcode{break}}

\pnum
\keyword{break} 语句应被\iref{stmt.pre}
\indextext{\idxgram{iteration-statement}}%
\indextext{statement!\idxcode{switch}}%
\grammarterm{iteration-statement}\iref{stmt.iter} 或 \keyword{switch} 语句\iref{stmt.switch} 包围。 \keyword{break} 语句导致终止最小的此类封闭语句;控制权传递到被终止语句之后的语句(如果有)。

\rSec2[stmt.cont]{ \keyword{continue} 语句}%
\indextext{statement!\idxcode{continue}}

\pnum
\keyword{continue} 语句应被\iref{stmt.pre}
\indextext{\idxgram{iteration-statement}}%
\grammarterm{iteration-statement}\iref{stmt.iter} 包围。\keyword{continue} 语句导致控制权传递到最小的此类封闭语句的循环继续部分,即循环的末尾。更准确地说,在每个语句中

\begin{minipage}{.30\hsize}
\begin{codeblock}
while (foo) {
  {
    // ...
  }
@\exposid{contin}@: ;
}
\end{codeblock}
\end{minipage}
\begin{minipage}{.30\hsize}
\begin{codeblock}
do {
  {
    // ...
  }
@\exposid{contin}@: ;
} while (foo);
\end{codeblock}
\end{minipage}
\begin{minipage}{.30\hsize}
\begin{codeblock}
for (;;) {
  {
    // ...
  }
@\exposid{contin}@: ;
}
\end{codeblock}
\end{minipage}

未包含在封闭迭代语句中的 \keyword{continue} 等价于 \tcode{goto} \exposid{contin}。

\rSec2[stmt.return]{ \keyword{return} 语句}%
\indextext{\idxcode{return}}%
\indextext{function return|see{\tcode{return}}}%

\pnum
函数通过 \tcode{return} 语句将控制权返回给其调用者。

\pnum
\tcode{return} 语句的 \grammarterm{expr-or-braced-init-list} 称为其操作数。没有操作数的 \tcode{return} 语句只能在返回类型为 \cv{}~\keyword{void} 的函数、构造函数\iref{class.ctor} 或析构函数\iref{class.dtor} 中使用。
\indextext{\idxcode{return}!constructor and}%
\indextext{\idxcode{return}!constructor and}%
操作数为 \keyword{void} 类型的 \tcode{return} 语句只能在具有 \cv{}~\keyword{void} 返回类型的函数中使用。具有任何其他操作数的 \tcode{return} 语句只能在返回类型不是 \cv{}~\keyword{void} 的函数中使用;
\indextext{conversion!return type}%
\tcode{return} 语句通过从操作数进行复制初始化\iref{dcl.init} 来初始化(显式或隐式)函数调用的返回引用或纯右值结果对象。
\begin{note}
构造函数或析构函数没有返回类型。
\end{note}
\begin{note}
如果操作数不是纯右值或其类型与函数的返回类型不同,则 \tcode{return} 语句可能涉及调用构造函数来执行操作数的复制或移动。如果返回自动存储期变量,则与 \tcode{return} 语句关联的复制操作可以被省略或转换为移动操作\iref{class.copy.elision}。
\end{note}

\pnum
结果对象的析构函数可能被调用\iref{class.dtor,except.ctor}。
\begin{example}
\begin{codeblock}
class A {
  ~A() {}
};
A f() { return A(); }   // 错误:\tcode{A} 的析构函数是私有的(即使它从未被调用)
\end{codeblock}
\end{example}

\pnum
流出构造函数、析构函数或具有 \cv{}~\keyword{void} 返回类型的非协程函数的末尾等价于没有操作数的 \tcode{return}。否则,流出既不是 \tcode{main}\iref{basic.start.main} 也不是协程\iref{dcl.fct.def.coroutine} 的函数的末尾会导致未定义的行为。

\pnum
对调用结果的复制初始化先序于由 \tcode{return} 语句的操作数建立的完整表达式末尾的临时对象的销毁,而后者又先序于包围 \tcode{return} 语句的块的局部变量\iref{stmt.jump} 的销毁。

\pnum
在返回类型为引用的函数中(除了 \tcode{std::is_convertible}\iref{meta.rel} 的虚构函数),将返回的引用绑定到临时表达式\iref{class.temporary} 的 \tcode{return} 语句是非良构的。
\begin{example}
\begin{codeblock}
auto&& f1() {
  return 42;            // 非良构
}
const double& f2() {
  static int x = 42;
  return x;             // 非良构
}
auto&& id(auto&& r) {
  return static_cast<decltype(r)&&>(r);
}
auto&& f3() {
  return id(42);        // OK, 但可能有错误
}
\end{codeblock}
\end{example}

\rSec2[stmt.return.coroutine]{ \keyword{co_return} 语句}%
\indextext{\idxcode{co_return}}%
\indextext{coroutine return|see{\tcode{co_return}}}%

\begin{bnf}
\nontermdef{coroutine-return-statement}\br
    \terminal{co_return} \opt{expr-or-braced-init-list} \terminal{;}
\end{bnf}

\pnum
\keyword{co_return} 语句将控制权转移到协程\iref{dcl.fct.def.coroutine}的调用者或恢复者。协程不应包含 \tcode{return} 语句\iref{stmt.return}。
\begin{note}
对于此判定,\tcode{return} 语句是否被 discarded statement\iref{stmt.if} 包围是无关紧要的。
\end{note}

\pnum
\keyword{co_return} 语句的 \grammarterm{expr-or-braced-init-list} 称为其操作数。令 \placeholder{p} 为命名协程承诺对象\iref{dcl.fct.def.coroutine} 的左值。\keyword{co_return} 语句等价于:
\begin{ncsimplebnf}
\terminal{\{} S\terminal{;} \terminal{goto} \exposid{final-suspend}\terminal{;} \terminal{\}}
\end{ncsimplebnf}
其中 \exposid{final-suspend} 是 \ref{dcl.fct.def.coroutine} 中定义的仅用于说明的标签,\placeholder{S} 定义如下:
\begin{itemize}
\item
如果操作数是 \grammarterm{braced-init-list} 或非 \keyword{void} 类型的表达式,则 \placeholder{S} 是 \placeholder{p}\tcode{.return_value(}\grammarterm{expr-or-braced-init-list}{}\tcode{)}。表达式 \placeholder{S} 应为 \keyword{void} 类型的纯右值。

\item
否则,\placeholder{S} 是 \grammarterm{compound-statement} \tcode{\{}{ }\opt{\grammarterm{expression}} \tcode{;} \placeholder{p}\tcode{.return_void()}\tcode{;{ }\}}。表达式 \placeholder{p}\tcode{.return_void()} 应为 \keyword{void} 类型的纯右值。
\end{itemize}

\pnum
如果在承诺类型的作用域中搜索名称 \tcode{return_void} 找到任何声明,则流出协程的 \grammarterm{function-body} 的末尾等价于没有操作数的 \keyword{co_return};否则流出协程的 \grammarterm{function-body} 的末尾会导致未定义的行为。

\rSec2[stmt.goto]{ \keyword{goto} 语句}%
\indextext{statement!\idxcode{goto}}

\pnum
\tcode{goto} 语句无条件地将控制权转移到由标识符标记的语句。标识符应为当前函数中的
\indextext{label}%
标签\iref{stmt.label}。

\rSec1[stmt.dcl]{声明语句}%
\indextext{statement!declaration}

\pnum
声明语句将一个或多个新名称引入块中;它的形式为
\begin{bnf}
\nontermdef{declaration-statement}\br
    block-declaration
\end{bnf}
\begin{note}
如果由声明引入的标识符先前在外层块中声明过,
\indextext{declaration hiding|see{name hiding}}%
\indextext{name hiding}%
\indextext{block (statement)!structure}%
则外部声明在该块的其余部分\iref{basic.lookup.unqual}被隐藏,之后它恢复其效力。
\end{note}

\pnum
\indextext{block (statement)!initialization in}%
\indextext{initialization!automatic}%
\indextext{active|see{variable, active}}%
具有自动存储期\iref{basic.stc.auto}的块变量在其 \grammarterm{init-declarator} 之后的所属作用域内的任何地方都是\defnx{活动的}{variable!active}。
\indextext{initialization!jump past}%
\indextext{\idxcode{goto}!initialization and}%
在函数内从点 $P$ 到点 $Q$ 的每次控制转移(包括语句的顺序执行)时,所有在 $P$ 处活动但不在 $Q$ 处活动的具有自动存储期的块变量将按其构造的相反顺序销毁。然后,所有在 $Q$ 处活动但不在 $P$ 处活动的具有自动存储期的块变量将按声明顺序初始化;除非所有此类变量都具有空初始化\iref{basic.life},否则控制转移不应是跳转。
\begin{footnote}
从 \keyword{switch} 语句的 condition 到 \keyword{case} 标签的转移在这方面被认为是跳转。
\end{footnote}
当执行 \grammarterm{declaration-statement} 时,$P$ 和 $Q$ 分别是其之前和之后的点;当函数返回时,$Q$ 在其主体之后。
\begin{example}
\begin{codeblock}
void f() {
  // ...
  goto lx;          // 错误:跳入 \tcode{a} 的作用域
  // ...
ly:
  X a = 1;
  // ...
lx:
  goto ly;          // OK,跳过意味着调用 \tcode{a} 的析构函数,然后在标签 \tcode{ly} 之后立即再次构造
}
\end{codeblock}
\end{example}

\pnum
\indextext{initialization!automatic}%
\indextext{initialization!dynamic block-scope}%
\indextext{initialization!local \tcode{static}}%
\indextext{initialization!local \tcode{thread_local}}%
具有静态存储期\iref{basic.stc.static}或线程存储期\iref{basic.stc.thread}的块变量的动态初始化在第一次控制通过其声明时执行;此类变量在其初始化完成后被认为是已初始化的。如果初始化通过抛出异常退出,则初始化未完成,因此下次控制进入声明时将再次尝试。如果当变量正在初始化时控制并发地进入声明,则并发执行应等待初始化完成。
\begin{note}
符合标准的实现不能在初始化程序的执行周围引入任何死锁。死锁仍可能由程序逻辑引起;实现只需要避免由于其自身的同步操作导致的死锁。
\end{note}
如果当变量正在初始化时控制递归地重新进入声明,则行为是未定义的。
\begin{example}
\begin{codeblock}
int foo(int i) {
  static int s = foo(2*i);      // 未定义行为:递归调用
  return i+1;
}
\end{codeblock}
\end{example}

\pnum
\indextext{\idxcode{static}!destruction of local}%
与具有静态或线程存储期的块变量相关联的对象当且仅当它被构造时才会被销毁。
\begin{note}
\ref{basic.start.term} 描述了此类对象被销毁的顺序。
\end{note}

\rSec1[stmt.ambig]{歧义消解}%
\indextext{ambiguity!declaration versus expression}

\pnum
在涉及 \grammarterm{expression-statement} 和 \grammarterm{declaration} 的语法中存在歧义:以函数式显式类型转换\iref{expr.type.conv} 作为其最左侧子表达式的 \grammarterm{expression-statement} 可能与第一个 \grammarterm{declarator} 以 \tcode{(} 开头的 \grammarterm{declaration} 无法区分。在这些情况下,\grammarterm{statement} 被认为是 \grammarterm{declaration},除非另有规定。

\pnum
\begin{note}
如果 \grammarterm{statement} 在语法上不能是 \grammarterm{declaration},则不存在歧义,因此该规则不适用。在某些情况下,需要检查整个 \grammarterm{statement} 来确定是否属于这种情况。这解决了许多示例的含义。
\begin{example}
假设 \tcode{T} 是 \grammarterm{simple-type-specifier}\iref{dcl.type.simple},

\begin{codeblock}
T(a)->m = 7;        // 表达式语句
T(a)++;             // 表达式语句
T(a,5)<<c;          // 表达式语句

T(*d)(int);         //  声明
T(e)[5];            //  声明
T(f) = { 1, 2 };    //  声明
T(*g)(double(3));   //  声明
\end{codeblock}

在上面的最后一个示例中,\tcode{g}(它是指向 \tcode{T} 的指针)被初始化为 \tcode{double(3)}。当然,由于语义原因,这是非良构的,但这不影响语法分析。
\end{example}

其余情况是 \grammarterm{declaration}。
\begin{example}
\begin{codeblock}
class T {
  // ...
public:
  T();
  T(int);
  T(int, int);
};
T(a);               //  声明
T(*b)();            //  声明
T(c)=7;             //  声明
T(d),e,f=3;         //  声明
extern int h;
T(g)(h,2);          //  声明
\end{codeblock}
\end{example}
\end{note}

\pnum
消歧纯粹是句法上的;也就是说,出现在此类语句中的名称的含义,除了它们是否是 \grammarterm{type-name} 之外,通常不会在消歧中使用或被消歧改变。类模板会根据需要进行实例化,以确定限定名是否为 \grammarterm{type-name}。消歧先于解析,并且被消歧为声明的语句可能是非良构的声明。如果在解析过程中,查找发现模板参数中的名称绑定到正在解析的声明(的一部分),则程序是非良构的。不需要诊断。
\begin{example}
\begin{codeblock}
struct T1 {
  T1 operator()(int x) { return T1(x); }
  int operator=(int x) { return x; }
  T1(int) { }
};
struct T2 { T2(int) { } };
int a, (*(*b)(T2))(int), c, d;

void f() {
  // 消歧要求将其解析为声明:
  T1(a) = 3,
  T2(4),                        // \tcode{T2} 将被声明为 \tcode{T1} 类型的变量,但这不会
  (*(*b)(T2(c)))(int(d));       // 允许声明的最后一部分正确解析,
                                // 因为它依赖于 \tcode{T2} 是一个类型名称
}
\end{codeblock}
\end{example}


\pnum
在语法上模棱两可的语句,如果它在语法上可以是一个最外层 \grammarterm{declarator} 带有 \grammarterm{trailing-return-type} 的 \grammarterm{declaration},那么只有当它以 \keyword{auto} 开头时,才被认为是 \grammarterm{declaration}。
\begin{example}
\begin{codeblock}
struct M;
struct S {
  S* operator()();
  int N;
  int M;

  void mem(S s) {
    auto(s)()->M;               // 表达式,\tcode{S::M} 隐藏了 \tcode{::M}
  }
};

void f(S s) {
  {
    auto(s)()->N;               // 表达式
    auto(s)()->M;               // 函数声明
  }
  {
    S(s)()->N;                  // 表达式
    S(s)()->M;                  // 表达式
  }
}
\end{codeblock}
\end{example}
\indextext{statement|)}
