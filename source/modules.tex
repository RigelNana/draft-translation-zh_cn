%!TEX root = std.tex
\rSec0[module]{模块}%

\gramSec[gram.module]{模块}

\rSec1[module.unit]{模块单元和 purview}

\begin{bnf}
\nontermdef{module-declaration}\br
    \opt{export-keyword} module-keyword module-name \opt{module-partition} \opt{attribute-specifier-seq} \terminal{;}
\end{bnf}

\begin{bnf}
\nontermdef{module-name}\br
    \opt{module-name-qualifier} identifier
\end{bnf}

\begin{bnf}
\nontermdef{module-partition}\br
    \terminal{:} \opt{module-name-qualifier} identifier
\end{bnf}

\begin{bnf}
\nontermdef{module-name-qualifier}\br
    identifier \terminal{.}\br
    module-name-qualifier identifier \terminal{.}
\end{bnf}

\pnum
\defn{模块单元}是包含 \grammarterm{module-declaration} 的翻译单元。\defnadj{命名}{module} 是具有相同 \grammarterm{module-name} 的模块单元的集合。\grammarterm{module-name} 或 \grammarterm{module-partition} 中不得出现标识符 \tcode{module} 和 \tcode{import} 作为 \grammarterm{identifier}。
\indextext{module!reserved name of}%
所有以由 \tcode{std} 后跟零个或多个 \grammarterm{digit} 组成的 \grammarterm{identifier} 开头或包含保留标识符\iref{lex.name} 的 \grammarterm{module-name} 都是保留的,不应在 \grammarterm{module-declaration} 中指定;不需要诊断。如果保留的 \grammarterm{module-name} 中的任何 \grammarterm{identifier} 是保留标识符,则该模块名称保留供 C++ 实现使用;否则它保留用于未来的标准化。可选的 \grammarterm{attribute-specifier-seq} 属于 \grammarterm{module-declaration}。

\pnum
\defn{模块接口单元}是其 \grammarterm{module-declaration} 以 \grammarterm{export-keyword} 开头的模块单元;任何其他模块单元都是\defn{模块实现单元}。命名模块应恰好包含一个没有 \grammarterm{module-partition} 的模块接口单元,称为该模块的\defn{主模块接口单元};不需要诊断。

\pnum
\defn{模块分区}是其 \grammarterm{module-declaration} 包含 \grammarterm{module-partition} 的模块单元。命名模块不应包含具有相同 \grammarterm{module-partition} 的多个模块分区。模块的所有作为模块接口单元的模块分区都应由主模块接口单元\iref{module.import}直接或间接导出。违反这些规则不需要诊断。
\begin{note}
模块分区只能由同一模块中的其他模块单元导入。模块划分为模块单元在模块外部不可见。
\end{note}

\pnum
\begin{example}
\begin{codeblocktu}{翻译单元 \#1}
export module A;
export import :Foo;
export int baz();
\end{codeblocktu}

\begin{codeblocktu}{翻译单元 \#2}
export module A:Foo;
import :Internals;
export int foo() { return 2 * (bar() + 1); }
\end{codeblocktu}

\begin{codeblocktu}{翻译单元 \#3}
module A:Internals;
int bar();
\end{codeblocktu}

\begin{codeblocktu}{翻译单元 \#4}
module A;
import :Internals;
int bar() { return baz() - 10; }
int baz() { return 30; }
\end{codeblocktu}

模块 \tcode{A} 包含四个翻译单元:
\begin{itemize}
\item 一个主模块接口单元,
\item 一个模块分区 \tcode{A:Foo},它是构成模块 \tcode{A} 接口一部分的模块接口单元,
\item 一个模块分区 \tcode{A:Internals},它不对模块 \tcode{A} 的外部接口做出贡献,以及
\item 一个提供 \tcode{bar} 和 \tcode{baz} 定义的模块实现单元,由于它没有分区名称,因此无法导入。
\end{itemize}
\end{example}

\pnum
\defnadj{模块单元}{purview} 是从 \grammarterm{module-declaration} 开始并延伸到翻译单元末尾的 \grammarterm{token} 序列。命名模块 \tcode{M} 的 \defnx{purview}{purview!named module} 是 \tcode{M} 的模块单元 purview 的集合。

\pnum
\defnadj{全局}{module} 是所有 \grammarterm{global-module-fragment} 和所有非模块单元的翻译单元的集合。出现在这种上下文中的声明被称为在全局模块的 \defnx{purview}{purview!global module} 中。
\begin{note}
全局模块没有名称,没有模块接口单元,也不是由任何 \grammarterm{module-declaration} 引入的。
\end{note}

\pnum
\defn{模块} 是命名模块或全局模块。声明\defnx{附加}{attached!declaration}到模块如下:
\begin{itemize}
\item
如果声明是一个非依赖的友元声明,它提名一个带有作为 \grammarterm{qualified-id} 或 \grammarterm{template-id} 的 \grammarterm{declarator-id} 的函数,或者提名一个不是带有既没有 \grammarterm{nested-name-specifier} 也没有 \grammarterm{simple-template-id} 的 \grammarterm{elaborated-type-specifier} 的类,则它附加到友元所附加到的模块\iref{basic.link}。
\item 否则,如果声明
\begin{itemize}
\item 是具有外部链接的 \grammarterm{namespace-definition} 或
\item 出现在 \grammarterm{linkage-specification}\iref{dcl.link} 中
\end{itemize}
则它附加到全局模块。

\item 否则,声明附加到它出现的模块的 purview 中。
\end{itemize}

\pnum
既不包含 \grammarterm{export-keyword} 也不包含 \grammarterm{module-partition} 的 \grammarterm{module-declaration} 隐式导入模块的主模块接口单元,就像通过 \grammarterm{module-import-declaration} 导入一样。
\begin{example}
\begin{codeblocktu}{翻译单元 \#1}
module B:Y;                     // 不隐式导入 \tcode{B}
int y();
\end{codeblocktu}

\begin{codeblocktu}{翻译单元 \#2}
export module B;
import :Y;                      // OK,不创建接口依赖循环
int n = y();
\end{codeblocktu}

\begin{codeblocktu}{翻译单元 \#3}
module B:X1;                    // 不隐式导入 \tcode{B}
int &a = n;                     // 错误:\tcode{n} 在此处不可见
\end{codeblocktu}

\begin{codeblocktu}{翻译单元 \#4}
module B:X2;                    // 不隐式导入 \tcode{B}
import B;
int &b = n;                     // OK
\end{codeblocktu}

\begin{codeblocktu}{翻译单元 \#5}
module B;                       // 隐式导入 \tcode{B}
int &c = n;                     // OK
\end{codeblocktu}
\end{example}

\rSec1[module.interface]{导出声明}%

\begin{bnf}
\nontermdef{export-declaration}\br
    \keyword{export} name-declaration\br
    \keyword{export} \terminal{\{} \opt{declaration-seq} \terminal{\}}\br
    export-keyword module-import-declaration
\end{bnf}

\pnum
\grammarterm{export-declaration} 应位于命名空间作用域中,并出现在模块接口单元的 purview 中。\grammarterm{export-declaration} 不应直接或间接出现在未命名命名空间或 \grammarterm{private-module-fragment} 中。 \grammarterm{export-declaration} 具有其 \grammarterm{name-declaration}、\grammarterm{declaration-seq}(如果有)或 \grammarterm{module-import-declaration} 的声明效果。 \grammarterm{export-declaration} 的 \grammarterm{name-declaration} 不应声明部分特化\iref{temp.decls.general}。 \grammarterm{export-declaration} 的 \grammarterm{declaration-seq} 不应包含 \grammarterm{export-declaration} 或 \grammarterm{module-import-declaration}。
\begin{note}
\grammarterm{export-declaration} 不建立作用域。
\end{note}

\pnum
如果声明在 \grammarterm{export-declaration} 中声明并位于命名空间作用域中,或者它是以下情况,则声明是\defnx{导出}{declaration!exported}的:
\begin{itemize}
\item 包含导出声明的 \grammarterm{namespace-definition},或
\item 头单元\iref{module.import}中引入至少一个名称的声明。
\end{itemize}

\pnum
如果导出的声明不在头单元中,则它不应声明具有内部链接的名称。

\pnum
\begin{example}
\begin{codeblocktu}{源文件 \tcode{"a.h"}}
export int x;
\end{codeblocktu}

\begin{codeblocktu}{翻译单元 \#1}
module;
#include "a.h"                  // 错误:\tcode{x} 的声明不在模块接口单元的 purview 中
export module M;
export namespace {}             // 错误:命名空间具有内部链接
namespace {
  export int a2;                // 错误:导出具有内部链接的名称
}
export static int b;            // 错误:b 显式声明为 static
export int f();                 // OK
export namespace N { }          // OK
export using namespace N;       // OK
\end{codeblocktu}
\end{example}

\pnum
如果导出的声明是 \grammarterm{using-declaration}\iref{namespace.udecl} 并且不在头单元中,则所有 \grammarterm{using-declarator} 最终引用的所有实体(如果有)都应已使用具有外部链接的名称引入。
\begin{example}
\begin{codeblocktu}{源文件 \tcode{"b.h"}}
int f();
\end{codeblocktu}

\begin{codeblocktu}{可导入头 \tcode{"c.h"}}
int g();
\end{codeblocktu}

\begin{codeblocktu}{翻译单元 \#1}
export module X;
export int h();
\end{codeblocktu}

\begin{codeblocktu}{翻译单元 \#2}
module;
#include "b.h"
export module M;
import "c.h";
import X;
export using ::f, ::g, ::h;     // OK
struct S;
export using ::S;               // 错误:\tcode{S} 具有模块链接
namespace N {
  export int h();
  static int h(int);            // \#1
}
export using N::h;              // 错误:\#1 具有内部链接
\end{codeblocktu}
\end{example}
\begin{note}
这些约束不适用于由 \keyword{typedef} 声明和 \grammarterm{alias-declaration} 引入的类型名称。
\begin{example}
\begin{codeblock}
export module M;
struct S;
export using T = S;             // OK,导出表示类型 \tcode{S} 的名称 \tcode{T}
\end{codeblock}
\end{example}
\end{note}

\pnum
如果实体 $X$ 是由导出的声明引入的,则实体 $X$ 的重声明被隐式导出;否则,如果它附加到命名模块,则不应导出。
\begin{example}
\begin{codeblock}
export module M;
struct S { int n; };
typedef S S;
export typedef S S;             // OK,不重声明实体
export struct S;                // 错误:导出的声明跟随非导出的声明
\end{codeblock}
\end{example}

\pnum
\begin{note}
由导出的声明引入的名称具有外部链接或没有链接;参见 \ref{basic.link}。由模块导出的命名空间作用域声明可以在导入该模块\iref{basic.lookup}的任何翻译单元中通过名称查找找到。类和枚举成员名称可以在类型的定义可到达的任何上下文中通过名称查找找到。
\end{note}
\begin{example}
\begin{codeblocktu}{\tcode{M} 的接口单元}
export module M;
export struct X {
  static void f();
  struct Y { };
};

namespace {
  struct S { };
}
export void f(S);               // OK
struct T { };
export T id(T);                 // OK

export struct A;                // 将 \tcode{A} 导出为不完整类型

export auto rootFinder(double a) {
  return [=](double x) { return (x + a/x)/2; };
}

export const int n = 5;         // OK, \tcode{n} 具有外部链接
\end{codeblocktu}

\begin{codeblocktu}{\tcode{M} 的实现单元}
module M;
struct A {
  int value;
};
\end{codeblocktu}

\begin{codeblocktu}{主程序}
import M;
int main() {
  X::f();                       // OK, \tcode{X} 被导出并且 \tcode{X} 的定义可到达
  X::Y y;                       // OK, \tcode{X::Y} 作为完整类型导出
  auto f = rootFinder(2);       // OK
  return A{45}.value;           // 错误:\tcode{A} 不完整
}
\end{codeblocktu}
\end{example}

\pnum
\begin{note}
导出的 \grammarterm{namespace-definition} 或导出的 \grammarterm{linkage-specification}\iref{dcl.link} 中的声明被导出并受导出声明规则的约束。
\begin{example}
\begin{codeblock}
export module M;
int g;
export namespace N {
  int x;                        // OK
  using ::g;                    // 错误:\tcode{::g} 具有模块链接
}
\end{codeblock}
\end{example}
\end{note}

\rSec1[module.import]{导入声明}%

\begin{bnf}
\nontermdef{module-import-declaration}\br
    import-keyword module-name \opt{attribute-specifier-seq} \terminal{;}\br
    import-keyword module-partition \opt{attribute-specifier-seq} \terminal{;}\br
    import-keyword header-name \opt{attribute-specifier-seq} \terminal{;}
\end{bnf}

\pnum
\grammarterm{module-import-declaration} 应位于全局命名空间作用域中。在模块单元中,所有 \grammarterm{module-import-declaration} 和导出 \grammarterm{module-import-declaration} 的 \grammarterm{export-declaration} 应出现在 \grammarterm{translation-unit} 和 \grammarterm{private-module-fragment}(如果有)的 \grammarterm{declaration-seq} 中的所有其他 \grammarterm{declaration} 之前。可选的 \grammarterm{attribute-specifier-seq} 属于 \grammarterm{module-import-declaration}。

\pnum
\grammarterm{module-import-declaration} \defnx{导入}{import} 一组翻译单元,具体确定如下所述。
\begin{note}
导入的翻译单元导出的命名空间作用域声明可以在导入翻译单元中通过名称查找\iref{basic.lookup}找到,并且导入翻译单元中的声明在导入声明之后在导入翻译单元中变为可到达的\iref{module.reach}。
\end{note}

\pnum
指定 \grammarterm{module-name} \tcode{M} 的 \grammarterm{module-import-declaration} 导入 \tcode{M} 的所有模块接口单元。

\pnum
指定 \grammarterm{module-partition} 的 \grammarterm{module-import-declaration} 只能出现在某个模块 \tcode{M} 的模块单元中的 \grammarterm{module-declaration} 之后。这样的声明导入 \tcode{M} 的同名模块分区。

\pnum
指定 \grammarterm{header-name} \tcode{H} 的 \grammarterm{module-import-declaration} 导入一个合成的 \defn{头单元},这是一个通过对 \tcode{H} 提名的源文件或头文件应用第 1 阶段到第 7 阶段的翻译\iref{lex.phases} 形成的翻译单元,其中不应包含 \grammarterm{module-declaration}。
\begin{note}
头单元是一个独立的翻译单元,具有一组独立的已定义宏。头单元中的所有声明都被隐式导出\iref{module.interface},并附加到全局模块\iref{module.unit}。
\end{note}
\defnadj{可导入}{header} 头文件是
\impldef{如何确定可导入头文件集}
头文件集的一个成员,该集合包括所有可导入的 C++ 库头文件\iref{headers}。\tcode{H} 应标识一个可导入的头文件。给定两个这样的 \grammarterm{module-import-declaration}:
\begin{itemize}
\item
如果它们的 \grammarterm{header-name} 标识不同的头文件或源文件\iref{cpp.include},则它们导入不同的头单元;
\item
否则,如果它们出现在同一个翻译单元中,则它们导入相同的头单元;
\item
否则,未指定它们是否导入相同的头单元。
\begin{note}
因此,在可导入头文件中声明的具有内部链接的实体可能存在多个副本。
\end{note}
\end{itemize}
\begin{note}
提名 \grammarterm{header-name} 的 \grammarterm{module-import-declaration} 也被预处理器识别,并导致在头单元的第 4 阶段翻译结束时定义的宏变为可见的,如 \ref{cpp.import} 中所述。任何其他 \grammarterm{module-import-declaration} 不会使宏可见。
\end{note}

\pnum
尽管所有声明都被隐式导出\iref{module.interface},但在头单元中允许声明具有内部链接的名称。
\begin{note}
出现在多个翻译单元中的定义通常不能引用此类名称\iref{basic.def.odr}。
\end{note}
头单元不应包含名称具有外部链接的非内联函数或变量的定义。


\pnum
当 \grammarterm{module-import-declaration} 导入翻译单元 $T$ 时,它还会导入由 $T$ 中导出的 \grammarterm{module-import-declaration} 导入的所有翻译单元;这样的翻译单元被称为由 $T$ \defnx{导出}{module!exported}。此外,当某个模块 $M$ 的模块单元中的 \grammarterm{module-import-declaration} 导入 $M$ 的另一个模块单元 $U$ 时,它还会导入由 $U$ 的模块单元 purview 中的非导出 \grammarterm{module-import-declaration} 导入的所有翻译单元。
\begin{footnote}
这与导入名称的查找规则\iref{basic.lookup}一致。
\end{footnote}
这些规则进而可能导致导入更多的翻译单元。
\begin{note}
这种间接导入不会使宏可用,因为翻译单元是翻译阶段 7\iref{lex.phases} 中的标记序列。可以通过直接导入头单元使宏可用,如 \ref{cpp.import} 中所述。
\end{note}

\pnum
模块实现单元不应被导出。
\begin{example}
\begin{codeblocktu}{翻译单元 \#1}
module M:Part;
\end{codeblocktu}

\begin{codeblocktu}{翻译单元 \#2}
export module M;
export import :Part;    // 错误:导出的分区 \tcode{:Part} 是一个实现单元
\end{codeblocktu}
\end{example}

\pnum
作为模块 \tcode{M} 的非模块分区的模块实现单元不应包含提名 \tcode{M} 的 \grammarterm{module-import-declaration}。
\begin{example}
\begin{codeblock}
module M;
import M;               // 错误:不能在其自己的单元中导入 \tcode{M}
\end{codeblock}
\end{example}

\pnum
如果一个翻译单元包含导入翻译单元 \tcode{U} 的声明(可能是 \grammarterm{module-declaration}),或者它对具有到 \tcode{U} 的接口依赖的翻译单元具有接口依赖,则该翻译单元对翻译单元 \tcode{U} 具有\defn{接口依赖}。翻译单元不应对自身具有接口依赖。
\begin{example}
\begin{codeblocktu}{\tcode{M1} 的接口单元}
export module M1;
import M2;
\end{codeblocktu}

\begin{codeblocktu}{\tcode{M2} 的接口单元}
export module M2;
import M3;
\end{codeblocktu}

\begin{codeblocktu}{\tcode{M3} 的接口单元}
export module M3;
import M1;              // 错误:循环接口依赖 $\mathtt{M3} \rightarrow \mathtt{M1} \rightarrow \mathtt{M2} \rightarrow \mathtt{M3}$
\end{codeblocktu}
\end{example}

\rSec1[module.global.frag]{全局模块片段}

\begin{bnf}
\nontermdef{global-module-fragment}\br
    module-keyword \terminal{;} \opt{declaration-seq}
\end{bnf}

\pnum
\begin{note}
在翻译的第 4 阶段之前,\grammarterm{declaration-seq}\iref{cpp.pre} 中只能出现预处理指令。
\end{note}

\pnum
\grammarterm{global-module-fragment} 指定模块单元的\defn{全局模块片段}的内容。全局模块片段可用于提供附加到全局模块并在模块单元内可用的声明。

\pnum
如果满足以下条件,则同一翻译单元中的声明 \tcode{D} 从声明 \tcode{S} \defn{声明可达}:
\begin{itemize}
\item
\tcode{D} 不声明函数或函数模板,并且 \tcode{S} 包含命名 \tcode{D} 的 \grammarterm{id-expression}、\grammarterm{namespace-name}、\grammarterm{type-name}、\grammarterm{template-name} 或 \grammarterm{concept-name},或

\item
\tcode{D} 声明一个由出现在 \tcode{S} 中的表达式\iref{basic.def.odr}命名的函数或函数模板,或

\item
\tcode{S} 包含依赖调用 \tcode{E}\iref{temp.dep},并且 \tcode{D} 通过为从 \tcode{E} 合成的表达式执行的任何名称查找找到,方法是将每个类型依赖的参数或操作数替换为没有关联命名空间或实体的占位符类型的值,或
\begin{note}
这包括在考虑重写 \tcode{!=} 表达式时执行的 \tcode{\keyword{operator}==} 的查找,在考虑重写关系比较时执行的 \tcode{\keyword{operator}<=>} 的查找,以及在考虑 \tcode{\keyword{operator}==} 是否为重写目标时执行的 \tcode{\keyword{operator}!=} 的查找。
\end{note}

\item
\tcode{S} 包含一个表达式,该表达式获取包含 \tcode{D} 且目标类型为依赖类型的重载集\iref{over.over}的地址,或

\item
存在一个不是 \grammarterm{namespace-definition} 的声明 \tcode{M},其中 \tcode{M} 从 \tcode{S} 声明可达,并且
\begin{itemize}
\item
\tcode{D} 从 \tcode{M} 声明可达,或
\item
\tcode{D} 和 \tcode{M} 声明相同的实体,并且 \tcode{D} 既不是友元声明也不位于块作用域中,或
\item
\tcode{D} 声明命名空间 \tcode{N} 并且 \tcode{M} 是 \tcode{N} 的成员,或
\item
\tcode{D} 和 \tcode{M} 之一声明类或类模板 \tcode{C},而另一个声明 \tcode{C} 的成员或友元,或
\item
\tcode{D} 和 \tcode{M} 之一声明枚举 \tcode{E},而另一个声明 \tcode{E} 的枚举器,或
\item
\tcode{D} 声明函数或变量,并且 \tcode{M} 在 \tcode{D} 中声明,
\begin{footnote}
声明可以出现在变量初始化器中的 \grammarterm{lambda-expression} 中。
\end{footnote}
或
\item
\tcode{D} 和 \tcode{M} 之一声明模板,而另一个声明该模板的部分或显式特化或隐式或显式实例化,或
\item
\tcode{M} 声明类模板,\tcode{D} 是该模板的推导指南,或
\item
\tcode{D} 和 \tcode{M} 之一声明类或枚举类型,而另一个为该类型引入用于链接目的的 typedef 名称。
\end{itemize}
\end{itemize}
在此确定中,未指定
\begin{itemize}
\item
是否在此确定之前将对 \grammarterm{alias-declaration}、\tcode{typedef} 声明、\grammarterm{using-declaration} 或 \grammarterm{namespace-alias-definition} 的引用替换为它们命名的声明,

\item
是否在此确定之前将不表示依赖类型且其 \grammarterm{template-name} 命名别名模板的 \grammarterm{simple-template-id} 替换为其表示的类型,

\item
是否在此确定之前将不表示依赖类型的 \grammarterm{decltype-specifier} 替换为其表示的类型,以及

\item
是否在此确定之前将非值依赖常量表达式替换为常量求值的结果。
\end{itemize}

\pnum
如果模块单元的全局模块片段中的声明 \tcode{D} 无法从 \grammarterm{translation-unit} 的 \grammarterm{declaration-seq} 中的任何 \grammarterm{declaration} 声明可达,则该声明 \tcode{D} 是\defnx{丢弃}{discarded!declaration}的。
\begin{note}
即使实例化上下文\iref{module.context}包括模块单元,丢弃的声明在模块单元外部也无法通过名称查找访问,也无法在实例化点\iref{temp.point}在模块单元外部的模板实例化中访问。
\end{note}

\pnum
\begin{example}
\begin{codeblock}
const int size = 2;
int ary1[size];                 // 未指定 \tcode{size} 是否从 \tcode{ary1} 声明可达
constexpr int identity(int x) { return x; }
int ary2[identity(2)];          // 未指定 \tcode{identity} 是否从 \tcode{ary2} 声明可达

template<typename> struct S;
template<typename, int> struct S2;
constexpr int g(int);

template<typename T, int N>
S<S2<T, g(N)>> f();             // \tcode{S}、\tcode{S2}、\tcode{g} 和 \tcode{::} 从 \tcode{f} 声明可达

template<int N>
void h() noexcept(g(N) == N);   // \tcode{g} 和 \tcode{::} 从 \tcode{h} 声明可达
\end{codeblock}
\end{example}

\pnum
\begin{example}
\begin{codeblocktu}{源文件 \tcode{"foo.h"}}
namespace N {
  struct X {};
  int d();
  int e();
  inline int f(X, int = d()) { return e(); }
  int g(X);
  int h(X);
}
\end{codeblocktu}

\begin{codeblocktu}{模块 \tcode{M} 接口}
module;
#include "foo.h"
export module M;
template<typename T> int use_f() {
  N::X x;                       // \tcode{N::X}、\tcode{N} 和 \tcode{::} 从 \tcode{use_f} 声明可达
  return f(x, 123);             // \tcode{N::f} 从 \tcode{use_f} 声明可达,
                                // \tcode{N::e} 从 \tcode{use_f} 间接声明可达,
                                //   因为它从 \tcode{N::f} 声明可达,并且
                                // \tcode{N::d} 从 \tcode{use_f} 声明可达,
                                //   因为它从 \tcode{N::f} 声明可达,
                                //   即使它没有在此调用中使用
}
template<typename T> int use_g() {
  N::X x;                       // \tcode{N::X}、\tcode{N} 和 \tcode{::} 从 \tcode{use_g} 声明可达
  return g((T(), x));           // \tcode{N::g} 无法从 \tcode{use_g} 声明可达
}
template<typename T> int use_h() {
  N::X x;                       // \tcode{N::X}、\tcode{N} 和 \tcode{::} 从 \tcode{use_h} 声明可达
  return h((T(), x));           // \tcode{N::h} 无法从 \tcode{use_h} 声明可达,但
                                // \tcode{N::h} 从 \tcode{use_h<int>} 声明可达
}
int k = use_h<int>();
  // \tcode{use_h<int>} 从 \tcode{k} 声明可达,因此
  // \tcode{N::h} 从 \tcode{k} 声明可达
\end{codeblocktu}

\begin{codeblocktu}{模块 \tcode{M} 实现}
module M;
int a = use_f<int>();           // OK
int b = use_g<int>();           // 错误:没有可用于调用 \tcode{g} 的可行函数;
                                // \tcode{g} 无法从模块 \tcode{M} 接口的 purview 声明可达,因此被丢弃
int c = use_h<int>();           // OK
\end{codeblocktu}
\end{example}

\rSec1[module.private.frag]{私有模块片段}

\begin{bnf}
\nontermdef{private-module-fragment}\br
    module-keyword \terminal{:} \keyword{private} \terminal{;} \opt{declaration-seq}
\end{bnf}

\pnum
\grammarterm{private-module-fragment} 只能出现在主模块接口单元\iref{module.unit}中。带有 \grammarterm{private-module-fragment} 的模块单元应是其模块的唯一模块单元;不需要诊断。


\pnum
\begin{note}
\grammarterm{private-module-fragment} 结束了模块接口单元中可能影响其他翻译单元行为的部分。 \grammarterm{private-module-fragment} 允许将模块表示为单个翻译单元,而无需使模块的所有内容对导入者都可达。 \grammarterm{private-module-fragment} 的存在会影响:
\begin{itemize}
\item
内联函数或变量的定义被要求的时点\iref{dcl.inline},

\item
具有占位符返回类型的导出函数的定义被要求的时点\iref{dcl.spec.auto},

\item
声明是否被要求不是暴露\iref{basic.link},

\item
内联函数和模板的定义必须出现的位置\iref{basic.def.odr,dcl.inline,temp.pre},

\item
在其之前实例化的模板的实例化上下文\iref{module.context},以及

\item
其中声明的可达性\iref{module.reach}。
\end{itemize}
\end{note}

\pnum
\begin{example}
\begin{codeblock}
export module A;
export inline void fn_e();      // 错误:导出的内联函数 \tcode{fn_e} 未在私有模块片段之前定义
inline void fn_m();             // 错误:未导出的内联函数 \tcode{fn_m} 未定义
static void fn_s();
export struct X;
export void g(X *x) {
  fn_s();                       // OK,调用同一翻译单元中的静态函数
}
export X *factory();            // OK

module :private;
struct X {};                    // \tcode{A} 的导入者无法访问的定义
X *factory() {
  return new X ();
}
void fn_e() {}
void fn_m() {}
void fn_s() {}
\end{codeblock}
\end{example}

\rSec1[module.context]{实例化上下文}

\pnum
\defn{实例化上下文} 是程序中的一组点,用于确定在特定声明或模板实例化的上下文中,哪些声明可以通过参数依赖名称查找\iref{basic.lookup.argdep}找到,哪些是可达的\iref{module.reach}。

\pnum
在默认函数\iref{special,class.compare.default}的隐式定义期间,实例化上下文是来自类定义的实例化上下文和导致默认函数隐式定义的程序构造的实例化上下文的并集。

\pnum
在模板的隐式实例化期间,如果其实例化点被指定为封闭特化\iref{temp.point}的实例化点,则实例化上下文是封闭特化的实例化上下文的并集,并且,如果模板在模块 $M$ 的模块接口单元中定义,并且实例化点不在 $M$ 的模块接口单元中,则为 $M$ 的主模块接口单元的 \grammarterm{declaration-seq} 末尾的点(在 \grammarterm{private-module-fragment} 之前,如果有)。

\pnum
在模板的隐式实例化期间,如果模板是因为在默认函数的隐式定义中被引用而隐式实例化的,则实例化上下文是默认函数的实例化上下文。

\pnum
在任何其他模板特化的实例化期间,实例化上下文包括模板的实例化点。

\pnum
在任何其他情况下,程序中某一点的实例化上下文包括该点。

\pnum
\begin{example}
\begin{codeblocktu}{翻译单元 \#1}
export module stuff;
export template<typename T, typename U> void foo(T, U u) { auto v = u; }
export template<typename T, typename U> void bar(T, U u) { auto v = *u; }
\end{codeblocktu}

\begin{codeblocktu}{翻译单元 \#2}
export module M1;
import "defn.h";        // 提供 \tcode{struct X \{\};}
import stuff;
export template<typename T> void f(T t) {
  X x;
  foo(t, x);
}
\end{codeblocktu}

\begin{codeblocktu}{翻译单元 \#3}
export module M2;
import "decl.h";        // 提供 \tcode{struct X;} (不是定义)
import stuff;
export template<typename T> void g(T t) {
  X *x;
  bar(t, x);
}
\end{codeblocktu}

\begin{codeblocktu}{翻译单元 \#4}
import M1;
import M2;
void test() {
  f(0);
  g(0);
}
\end{codeblocktu}
对 \tcode{f(0)} 的调用有效;\tcode{foo<int, X>} 的实例化上下文包括
\begin{itemize}
\item 翻译单元 \#1 的末尾的点,
\item 翻译单元 \#2 的末尾的点,以及
\item 对 \tcode{f(0)} 的调用的点,
\end{itemize}
因此 \tcode{X} 的定义是可达的\iref{module.reach}。

未指定对 \tcode{g(0)} 的调用是否有效:\tcode{bar<int, X>} 的实例化上下文包括
\begin{itemize}
\item 翻译单元 \#1 的末尾的点,
\item 翻译单元 \#3 的末尾的点,以及
\item 对 \tcode{g(0)} 的调用的点,
\end{itemize}
因此 \tcode{X} 的定义不一定是可达的,如 \ref{module.reach} 中所述。
\end{example}

\rSec1[module.reach]{可达性}

\indextext{necessarily reachable|see{reachable, necessarily}}
\pnum
如果翻译单元 $U$ 是包含 $P$ 的翻译单元具有接口依赖的模块接口单元,或者包含 $P$ 的翻译单元在 $P$\iref{module.import} 之前导入 $U$,则翻译单元 $U$ 从点 $P$ \defnx{必然可达}{reachable!necessarily!translation unit}。
\begin{note}
虽然模块接口单元即使仅通过非导出的导入声明传递导入也是可达的,但来自此类模块接口单元的命名空间作用域名称无法通过名称查找\iref{basic.lookup}找到。
\end{note}

\pnum
所有必然可达的翻译单元都是\defnx{可达}{reachable!translation unit}的。程序中的点具有接口依赖的其他翻译单元可以被认为是可达的,但未指定哪些以及在什么情况下可达。
\begin{footnote}
因此,不要求实现防止观察到编译中涉及的其他翻译单元的语义影响。
\end{footnote}
\begin{note}
建议在打算可移植的程序中避免依赖任何其他翻译单元的可达性。
\end{note}

\pnum
如果满足以下条件,则声明 $D$ 从点 $P$ \defnx{可达}{reachable from!declaration}:
\begin{itemize}
\item $D$ 出现在同一翻译单元中的 $P$ 之前,或者
\item $D$ 未被丢弃\iref{module.global.frag},出现在从 $P$ 可达的翻译单元中,并且不出现在 \grammarterm{private-module-fragment} 中。
\end{itemize}
如果声明可以从实例化上下文\iref{module.context}中的任何点访问,则该声明是\defnx{可达}{reachable!declaration}的。
\begin{note}
声明是否导出与其是否可达无关。
\end{note}

\pnum
上下文中实体的所有可达声明的累积属性决定了该实体在该上下文中的行为。
\begin{note}
这些可达的语义属性包括类型完整性、类型定义、初始化器、函数或模板声明的默认参数、属性、绑定的名称等。由于默认参数在调用表达式的上下文中求值,因此相应参数类型的可达语义属性在该上下文中适用。
\begin{example}
\begin{codeblocktu}{翻译单元 \#1}
export module M:A;
export struct B;
\end{codeblocktu}

\begin{codeblocktu}{翻译单元 \#2}
module M:B;
struct B {
  operator int();
};
\end{codeblocktu}

\begin{codeblocktu}{翻译单元 \#3}
module M:C;
import :A;
B b1;                           // 错误:\tcode{struct B} 没有可达定义
\end{codeblocktu}

\begin{codeblocktu}{翻译单元 \#4}
export module M;
export import :A;
import :B;
B b2;
export void f(B b = B());
\end{codeblocktu}

\begin{codeblocktu}{翻译单元 \#5}
import M;
B b3;                           // 错误:\tcode{struct B} 没有可达定义
void g() { f(); }               // 错误:\tcode{struct B} 没有可达定义
\end{codeblocktu}
\end{example}
\end{note}

\pnum
\begin{note}
即使无法通过名称查找找到实体的声明,它们也可以是可达的。
\end{note}
\begin{example}
\begin{codeblocktu}{翻译单元 \#1}
export module A;
struct X {};
export using Y = X;
\end{codeblocktu}

\begin{codeblocktu}{翻译单元 \#2}
import A;
Y y;                // OK, \tcode{X} 的定义是可达的
X x;                // 错误:\tcode{X} 对非限定查找不可见
\end{codeblocktu}
\end{example}
