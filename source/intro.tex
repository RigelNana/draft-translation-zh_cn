%!TEX root = std.tex

\rSec0[intro.scope]{适用范围}

\pnum
\indextext{scope|(}%
本标准规定了C++编程语言实现的要求。首要要求是任何实现必须正确实现该语言,因此本标准也定义了C++语言。在本标准其他章节中,对首要要求的补充说明和放宽条件将散见于各处。

\pnum
\Cpp{}是基于\IsoC{}所描述的C编程语言发展而来的通用编程语言。相较于C语言,\Cpp{}提供了诸多扩展功能,包括额外的数据类型、类、模板、异常处理、命名空间、运算符重载、函数名重载、引用、自由存储管理运算符,以及更多库功能。%
\indextext{scope|)}

\rSec0[intro.refs]{规范性引用}%
\indextext{normative references|see{引用,规范性}}%

\pnum
\indextext{references!normative|(}%
以下文件中的条款通过本文的引用而成为本标准的条款。凡是注日期的引用文件,仅所注日期的版本适用于本标准。凡是不注日期的引用文件,其最新版本(包括所有的修改单)适用于本标准。
\begin{itemize}
% ISO文档按编号排序
\item ISO/IEC 2382,\doccite{信息技术 词汇}
\item ISO 8601-1:2019,\doccite{日期与时间 信息交换表示法 第1部分:基本规则}
\item \IsoC{},\doccite{信息技术 编程语言 C}
\item \IsoPosix{},\doccite{信息技术 可移植操作系统接口(POSIX\textregistered)\begin{footnote}
POSIX\textregistered\ 是电气与电子工程师学会的注册商标。
此信息仅为本文档使用者提供便利,不构成ISO或IEC对该产品的认可。
\end{footnote}
基础规范 第7版}
\item \IsoPosix{}/Cor 1:2013,\doccite{信息技术 可移植操作系统接口(POSIX\textregistered)基础规范 第7版 技术勘误1}
\item \IsoPosix{}/Cor 2:2017,\doccite{信息技术 可移植操作系统接口(POSIX\textregistered)基础规范 第7版 技术勘误2}
\item \IsoFloatUndated{}:2020,\doccite{信息技术 微处理器系统 浮点算术}
\item ISO 80000-2:2019,\doccite{量和单位 第2部分:数学}
% 其他国际标准
\item Ecma International,\doccite{ECMAScript
\begin{footnote}
ECMAScript\textregistered\ 是Ecma International的注册商标。
此信息仅为本文档使用者提供便利,不构成ISO或IEC对该产品的认可。
\end{footnote}
语言规范},Ecma-262标准第三版,1999年。
\item
Unicode联盟,\doccite{Unicode标准}。
访问地址:\url{https://www.unicode.org/versions/latest/}
\end{itemize}
\indextext{references!normative|)}

\rSec0[intro.defs]{术语和定义}

\pnum
\indextext{definitions|(}%
就本文档而言,
适用 ISO/IEC 2382、ISO 80000-2:2019
以及下列各项中给出的术语和定义。

\pnum
ISO 和 IEC 维护用于标准化的术语数据库,
地址如下:
\begin{itemize}
\item ISO 在线浏览平台:可在 \url{https://www.iso.org/obp} 获取
\item IEC Electropedia:可在 \url{https://www.electropedia.org/} 获取
\end{itemize}

\indexdefn{access}%
\definition{访问}{defns.access}
\defncontext{执行时动作}
读取或修改对象的值

\begin{defnote}
只有标量类型的泛左值才能用于访问对象。
标量对象的读取在 \ref{conv.lval} 中描述,
标量对象的修改在
\ref{expr.ass}、\ref{expr.post.incr} 和 \ref{expr.pre.incr} 中描述。
尝试读取或修改类类型的对象
通常会调用构造函数\iref{class.ctor}
或赋值运算符\iref{class.copy.assign};
此类调用本身并不构成访问,
尽管它们可能涉及标量子对象的访问。
\end{defnote}

\indexdefn{argument}%
\indexdefn{argument!function call expression}
\definition{实参}{defns.argument}
\defncontext{函数调用表达式} 括号括起来的逗号分隔列表中的表达式

\indexdefn{argument}%
\indexdefn{argument!function-like macro}%
\definition{实参}{defns.argument.macro}
\defncontext{类函数宏} 括号括起来的逗号分隔列表中的预处理记号序列

\indexdefn{argument}%
\indexdefn{argument!throw expression}%
\definition{实参}{defns.argument.throw}
\defncontext{throw 表达式} \keyword{throw} 的操作数

\indexdefn{argument}%
\indexdefn{argument!template instantiation}%
\definition{实参}{defns.argument.templ}
\defncontext{模板实例化}
尖括号括起来的逗号分隔列表中的
\grammarterm{常量表达式}、
\grammarterm{类型标识} 或
\grammarterm{标识表达式}

\indexdefn{block (execution)}%
\definition{阻塞}{defns.block}
\defncontext{执行}
等待某个条件(除了等待实现执行该执行线程的执行步骤之外)满足,
然后再继续执行到阻塞操作之后

\indexdefn{block (statement)}%
\definition{块}{defns.block.stmt}
\defncontext{语句}
复合语句

\indexdefn{C!standard library}%
\definition{C标准库}{defns.c.lib}
\IsoC{} 第 7 条款中描述的库
\begin{defnote}
根据 \ref{\firstlibchapter}
到 \ref{\lastlibchapter} 以及 \ref{diff.library} 中指出的限定条件,
C 标准库是 \Cpp{} 标准库的子集。
\end{defnote}

\definition{字符}{defns.character}
\indexdefn{character}%
\defncontext{库}
当被顺序处理时,
可以表示文本的对象

\begin{defnote}
该术语不仅仅指
\tcode{char}、
\keyword{char8_t}、
\keyword{char16_t}、
\keyword{char32_t}、
和
\keyword{wchar_t}
对象\iref{basic.fundamental},
还指任何可以用一个类型表示的值,
该类型提供了在
\ref{strings}、\ref{localization}、\ref{input.output} 或~\ref{re} 中指定的定义。
\end{defnote}

\definition{字符容器类型}{defns.character.container}
\defncontext{库}
\indexdefn{type!character container}%
用于表示 \termref{defns.character}{字符}{} 的类或类型

\begin{defnote}
它被用作 \tcode{char_traits}
和使用它的类模板(
如 string、iostream 和regex类模板)的模板参数之一。
\end{defnote}

\definition{测序元素}{defns.regex.collating.element}
\indexdefn{collating element}%
当前 locale 中一个或多个 \termref{defns.character}{字符}{} 的序列,
它们被视为单个字符进行排序

\definition{组件}{defns.component}
\defncontext{库}
\indexdefn{component}%
作为成员、\termref{defns.parameter}{形参}{} 或
返回类型直接相关的一组库实体

\begin{defnote}
例如,类模板 \tcode{basic_string}
和对字符串进行操作的非成员函数模板
被称为字符串组件。
\end{defnote}

\indexdefn{behavior!conditionally-supported}%
\definition{条件支持}{defns.cond.supp}
不要求实现支持的程序结构

\begin{defnote}
每个实现都记录了它不支持的所有条件支持的
结构。
\end{defnote}

\definition{常量求值}{defns.const.eval}
\indexdefn{constant evaluation}%
作为核心常量表达式\iref{expr.const}求值表达式的一部分而执行的求值

\definition{常量子表达式}{defns.const.subexpr}
\indexdefn{constant subexpression}%
作为
\grammarterm{条件表达式}
\tcode{CE} 的子表达式进行求值时,
不会阻止 \tcode{CE}
成为核心常量表达式的表达式

\definition{死锁}{defns.deadlock}
\defncontext{库}
\indexdefn{deadlock}%
一种情形,其中一个或多个线程无法继续执行,因为每个线程都在
\termref{defns.block}{阻塞}{}等待其他一个或多个线程满足某个条件

\definition{默认行为}{defns.default.behavior.impl}
\indexdefn{behavior!default}%
\defncontext{库实现}
在 \termref{defns.required.behavior}{要求行为}{} 范围内,由实现提供的特定行为

\indexdefn{message!diagnostic}%
\definition{诊断消息}{defns.diagnostic}
属于实现输出消息中 \impldef{诊断消息} 子集的消息

\indexdefn{type!dynamic}%
\definition{动态类型}{defns.dynamic.type}
\defncontext{泛左值} 泛左值所指的最底层派生对象的类型

\begin{example}
如果一个静态类型为“指向类 \tcode{B} 的指针”\iref{dcl.ptr} 的指针 \tcode{p} 指向一个从 \tcode{B}\iref{class.derived} 派生的类 \tcode{D} 的对象,则表达式 \tcode{*p} 的动态类型为“\tcode{D}”。引用\iref{dcl.ref} 的处理方式类似。
\end{example}

\indexdefn{type!dynamic}%
\definition{动态类型}{defns.dynamic.type.prvalue}
\defncontext{纯右值} 纯右值表达式的 \termref{defns.static.type}{静态类型}{}

\indexdefn{behavior!erroneous}%
\definition{错误行为}{defns.erroneous}
建议实现进行诊断的良定行为
\begin{defnote}
错误行为总是由不正确的程序代码导致的。
允许但不要求实现对其进行诊断\iref{intro.compliance.general}。
常量表达式\iref{expr.const}的求值永远不会表现出 \ref{intro} 到 \ref{\lastcorechapter} 中指定的错误行为。
\end{defnote}

\definition{表达式等价}{defns.expression.equivalent}
\defncontext{库}
\indexdefn{expression-equivalent}%
具有相同效果的表达式,
它们要么都是潜在抛出的,
要么都不是潜在抛出的,
并且
要么都是 \termref{defns.const.subexpr}{常量子表达式}{},
要么都不是常量子表达式

\begin{example}
对于类型为 \tcode{int} 的值 \tcode{x}
和一个接受整数参数的函数 \tcode{f},
表达式
\tcode{f(x + 2)}、
\tcode{f(2 + x)}
和
\tcode{f(1 + x + 1)}
是表达式等价的。
\end{example}

\definition{有限状态机}{defns.regex.finite.state.machine}
\defncontext{正则表达式}
\indexdefn{finite state machine}%
一种未指定的数据结构,用于
表示 \termref{defns.regex.regular.expression}{正则表达式}{},并且允许
高效地获得与正则表达式的匹配

\definition{格式说明符}{defns.regex.format.specifier}
\defncontext{正则表达式}
\indexdefn{format specifier}%
一个或多个 \termref{defns.character}{字符}{} 的序列,预期
被 \termref{defns.regex.regular.expression}{正则表达式}{} 匹配的某些部分替换

\definition{处理函数}{defns.handler}
\defncontext{库}
\indexdefn{function!handler}%
可以由 \Cpp{} 程序提供定义的非保留函数

\begin{defnote}
\Cpp{} 程序可以通过
在调用任何安装处理函数的库函数时提供指向该函数的指针,
在其执行的各个点指定处理函数(参见 \ref{support})。
\end{defnote}

\indexdefn{program!ill-formed}%
\definition{病构程序}{defns.ill.formed}
不是良构的\iref{defns.well.formed}程序

\indexdefn{behavior!implementation-defined}%
\definition{实现定义行为}{defns.impl.defined}
对于 \termref{defns.well.formed}{良构程序}{} 结构和正确的数据,
依赖于实现并且每个实现都记录在案的行为

\definition{指针上的实现定义严格全序}{defns.order.ptr}
\indexdefn{pointer!strict total order}%
\defncontext{库}
\impldef{指针值上的严格全序}
所有指针值上的严格全序,
使得该排序与内置运算符
\tcode{<}、\tcode{>}、\tcode{<=}、\tcode{>=} 和 \tcode{<=>}
所施加的偏序一致

\indexdefn{limits!implementation}%
\definition{实现限制}{defns.impl.limits}
实现对程序施加的限制

\indexdefn{behavior!locale-specific}%
\definition{区域特定行为}{defns.locale.specific}
依赖于每个实现都记录在案的国籍、文化和
语言的本地约定的行为

\definition{匹配}{defns.regex.matched}
\defncontext{正则表达式}
\indexdefn{matched}%
\indexdefn{regular expression!matched}%
零个或多个 \termref{defns.character}{字符}{} 的序列
对应于由模式定义的字符序列时的条件

\definition{修改器函数}{defns.modifier}
\defncontext{库}
\indexdefn{function!modifier}%
改变类对象状态的、
除构造函数、
赋值运算符或析构函数之外的
类成员函数

\definition{移动赋值}{defns.move.assign}
\defncontext{库}
\indexdefn{assignment!move}%
将某个对象类型的右值赋值给同一类型的可修改左值

\definition{移动构造}{defns.move.constr}
\defncontext{库}
\indexdefn{construction!move}%
使用相同类型的右值对某个类型的对象进行直接初始化

\indexdefn{library call!non-constant}%
\definition{非常量库调用}{defns.nonconst.libcall}
作为任何表达式 \tcode{E} 求值的一部分,
阻止 \tcode{E} 成为核心常量表达式的库函数调用

\definition{NTCTS}{defns.ntcts}
\defncontext{库}
\indexdefn{NTCTS}%
\indexdefn{string!null-terminated character type}%
具有
\termref{defns.character}{字符}{} 类型
且位于终止空字符类型
值 \tcode{charT()} 之前的值序列

\definition{观察者函数}{defns.observer}
\defncontext{库}
\indexdefn{function!observer}%
访问类对象状态
但不改变该状态的类成员函数

\begin{defnote}
观察者函数被指定为
\keyword{const}
成员函数。
\end{defnote}

\indexdefn{parameter}%
\indexdefn{parameter!function}%
\indexdefn{parameter!catch clause}%
\definition{形参}{defns.parameter}
\defncontext{函数或 catch 子句} 在函数声明或定义中,或在异常处理程序的 catch 子句中声明的
在进入函数或处理程序时获取值的对象或引用

\indexdefn{parameter}%
\indexdefn{parameter!function-like macro}%
\definition{形参}{defns.parameter.macro}
\defncontext{类函数宏} 紧跟在宏名称后面的括号括起来的逗号分隔列表中的标识符

\indexdefn{parameter}%
\indexdefn{parameter!template}%
\definition{形参}{defns.parameter.templ}
\defncontext{模板} \grammarterm{模板参数列表} 的成员

\definition{主等价类}{defns.regex.primary.equivalence.class}
\defncontext{正则表达式}
\indexdefn{primary equivalence class}%
共享相同主排序键的一个或多个 \termref{defns.character}{字符}{} 的集合:即仅依赖于字符形状,
而不依赖于重音、大小写或区域特定定制的排序键权重

\definition{程序定义的特化}{defns.prog.def.spec}
\defncontext{库}
\indexdefn{specialization!program-defined}%
不是 \Cpp{} 标准库的一部分
且不是由实现定义的显式模板特化或部分特化

\definition{程序定义的类型}{defns.prog.def.type}
\defncontext{库}
\indexdefn{type!program-defined}%
不是 \Cpp{} 标准库的一部分且不是由实现定义的非闭包类类型或枚举类型,
或非实现提供的 lambda 表达式的闭包类型,
或 \termref{defns.prog.def.spec}{程序定义的特化}{} 的实例化

\begin{defnote}
由实现定义的类型包括
扩展\iref{intro.compliance} 和库使用的内部类型。
\end{defnote}

\definition{投影}{defns.projection}
\indexdefn{projection}%
\defncontext{库}
算法在检查元素值之前应用的变换

\begin{example}
\begin{codeblock}
std::pair<int, std::string_view> pairs[] = {{2, "foo"}, {1, "bar"}, {0, "baz"}};
std::ranges::sort(pairs, std::ranges::less{}, [](auto const& p) { return p.first; });
\end{codeblock}
按其 \tcode{first} 成员的升序对 pairs 进行排序:
\begin{codeblock}
{{0, "baz"}, {1, "bar"}, {2, "foo"}}
\end{codeblock}
\end{example}

\definition{可引用类型}{defns.referenceable}
\indexdefn{type!referenceable}%
对象类型、
不具有 cv 限定符或 \grammarterm{ref-qualifier} 的函数类型,
或引用类型

\begin{defnote}
该术语描述了可以创建引用的类型,
包括引用类型。
\end{defnote}

\definition{正则表达式}{defns.regex.regular.expression}
从一组 \termref{defns.character}{字符}{} 串中选择特定字符串的模式

\definition{替换函数}{defns.replacement}
\defncontext{库}
\indexdefn{function!replacement}%
由 \Cpp{} 程序提供定义的非保留函数

\begin{defnote}
在程序执行期间,由于创建程序\iref{lex.phases} 和解析
所有翻译单元\iref{basic.link} 的定义,
对于这样的函数,只有一个定义有效。
\end{defnote}

\definition{要求行为}{defns.required.behavior}
\defncontext{库}
\indexdefn{behavior!required}%
适用于实现提供的行为以及程序中任何此类函数定义的行为的
\termref{defns.replacement}{替换函数}{} 和
\termref{defns.handler}{处理函数}{} 语义的描述

\begin{defnote}
如果 \Cpp{} 程序中定义的此类函数在执行时未能满足要求行为,
则该行为是未定义的。%
\indextext{undefined}
\end{defnote}

\definition{保留函数}{defns.reserved.function}
\defncontext{库}
\indexdefn{function!reserved}%
作为 \Cpp{} 标准库的一部分指定的、
由实现定义的函数

\begin{defnote}
如果 \Cpp{} 程序为任何保留函数提供了定义,则结果是未定义的。%
\indextext{undefined}
\end{defnote}

\definition{运行时未定义行为}{defns.undefined.runtime}
\indexdefn{behavior!runtime-undefined}%
除常量求值期间发生外均为未定义的行为

\begin{defnote}
在常量求值期间,
\begin{itemize}
\item
\impldef{运行时未定义行为是否导致表达式被视为非常量}
运行时未定义行为是否导致表达式被视为非常量(如 \ref{expr.const} 中所述)
\item
运行时未定义行为没有其他影响。
\end{itemize}
\end{defnote}

\indexdefn{signature}%
\definition{签名}{defns.signature}
\defncontext{函数}
名称、
形参类型列表 (parameter-type-list)
和外围命名空间

\begin{defnote}
签名被用作
名称重整和链接的基础。
\end{defnote}

\indexdefn{signature}%
\definition{签名}{defns.signature.friend}
\defncontext{带有尾随 \grammarterm{requires-clause} 的非模板友元函数}
名称、
形参类型列表、
外围类、
和
尾随 \grammarterm{requires-clause}

\indexdefn{signature}%
\definition{签名}{defns.signature.templ}
\defncontext{函数模板}
名称、
形参类型列表、
外围命名空间、
返回类型、
\grammarterm{template-head} 的 \termref{defns.signature.template.head}{签名}{}、
和
尾随 \grammarterm{requires-clause}(如果有)

\indexdefn{signature}%
\definition{签名}{defns.signature.templ.friend}
\defncontext{带有涉及外围模板形参的约束的友元函数模板}
名称、
形参类型列表、
返回类型、
外围类、
\grammarterm{template-head} 的 \termref{defns.signature.template.head}{签名}{}、
和
尾随 \grammarterm{requires-clause}(如果有)

\indexdefn{signature}%
\definition{签名}{defns.signature.spec}
\defncontext{函数模板特化} 它是其特化的模板的 \termref{defns.signature.templ}{签名}{}
及其模板 \termref{defns.argument.templ}{实参}{}(无论是显式指定的还是推导的)

\indexdefn{signature}%
\definition{签名}{defns.signature.member}
\defncontext{类成员函数}
名称、
形参类型列表、
函数所属的类、
cv 限定符(如果有)、
\grammarterm{ref-qualifier}(如果有)、
和
尾随 \grammarterm{requires-clause}(如果有)

\indexdefn{signature}%
\definition{签名}{defns.signature.member.templ}
\defncontext{类成员函数模板}
名称、
形参类型列表、
函数所属的类、
cv 限定符(如果有)、
\grammarterm{ref-qualifier}(如果有)、
返回类型(如果有)、
\grammarterm{template-head} 的 \termref{defns.signature.template.head}{签名}{}、
和
尾随 \grammarterm{requires-clause}(如果有)

\indexdefn{signature}%
\definition{签名}{defns.signature.member.spec}
\defncontext{类成员函数模板特化} 它是其特化的成员函数模板的 \termref{defns.signature.member.templ}{签名}{}
及其模板实参(无论是显式指定的还是推导的)

\indexdefn{signature}%
\definition{签名}{defns.signature.template.head}
\defncontext{\grammarterm{template-head}}
模板 \termref{defns.parameter.templ}{形参}{} 列表,
不包括模板形参名称和默认 \termref{defns.argument.templ}{实参}{},
和
\grammarterm{requires-clause}(如果有)

\definition{稳定算法}{defns.stable}
\defncontext{库}
\indexdefn{algorithm!stable}%
\indexdefn{stable algorithm}%
根据特定算法的需要,保持元素顺序的算法

\begin{defnote}
稳定算法的要求在 \ref{algorithm.stable} 中给出。
\end{defnote}

\indexdefn{type!static}%
\definition{静态类型}{defns.static.type}
不考虑执行语义,通过分析程序得到的表达式的类型

\begin{defnote}
表达式的静态类型仅取决于
表达式出现的程序形式,并且在程序执行期间不会更改。
\end{defnote}

\definition{子表达式}{defns.regex.subexpression}
\defncontext{正则表达式}
\indexdefn{sub-expression!regular expression}%
已用括号标记的 \termref{defns.regex.regular.expression}{正则表达式}{} 的子集

\definition{特征类}{defns.traits}
\defncontext{库}
\indexdefn{traits}%
封装了一组类型和函数的类,这些类型和函数是类模板和
函数模板操作实例化对象类型所必需的

\indexdefn{unblock}%
\definition{解除阻塞}{defns.unblock}
满足一个或多个 \termref{defns.block}{阻塞}{} 执行线程正在等待的条件

\indexdefn{behavior!undefined}%
\definition{未定义行为}{defns.undefined}
本文档
对此未作任何要求的行为

\begin{defnote}
当本文档省略任何显式
行为定义,或当程序使用不正确的结构或无效数据时,可能会出现未定义行为。
允许的未定义行为范围
从完全忽略这种情况并产生不可预测的结果,到
在翻译或程序执行期间以环境特征的文档化方式运行
(无论是否发出 \termref{defns.diagnostic}{诊断消息}{}),到终止翻译或执行(发出
诊断消息)。许多不正确的程序结构并
不会导致未定义行为;它们需要被诊断。
常量表达式\iref{expr.const}的求值永远不会表现出 \ref{intro} 到 \ref{\lastcorechapter} 中明确指定的未定义行为。
\end{defnote}

\indexdefn{behavior!unspecified}%
\definition{未指明行为}{defns.unspecified}
对于 \termref{defns.well.formed}{良构程序}{} 结构和正确的数据,
依赖于实现的行为

\begin{defnote}
不要求实现
记录发生哪种行为。
可能的行为范围通常由本文档划定。
\end{defnote}

\definition{有效但未指明状态}{defns.valid}
\defncontext{库}
\indexdefn{valid but unspecified state}%
未指定的对象值,但对象的
不变量得到满足,并且对对象的操作按照其类型指定的方式进行

\begin{example}
如果类型为 \tcode{std::vector<int>} 的对象 \tcode{x} 处于
有效但未指明状态,则可以无条件调用 \tcode{x.empty()},
并且只有当 \tcode{x.empty()} 返回
\tcode{false} 时才能调用 \tcode{x.front()}。
\end{example}

\indexdefn{program!well-formed}%
\definition{良构程序}{defns.well.formed}
根据语法和语义规则构造的 \Cpp{} 程序
\indextext{definitions|)}

\rSec0[intro]{一般原则}

\rSec1[intro.compliance]{实现符合性}%
\indextext{diagnostic message|see{message, diagnostic}}%
\indexdefn{conditionally-supported behavior|see{behavior, con\-ditionally-supported}}%
\indextext{dynamic type|see{type, dynamic}}%
\indextext{static type|see{type, static}}%
\indextext{ill-formed program|see{program, ill-formed}}%
\indextext{well-formed program|see{program, well-formed}}%
\indextext{implementation-defined behavior|see{behavior, im\-plementation-defined}}%
\indextext{undefined behavior|see{behavior, undefined}}%
\indextext{unspecified behavior|see{behavior, unspecified}}%
\indextext{implementation limits|see{limits, implementation}}%
\indextext{locale-specific behavior|see{behavior, locale-spe\-cific}}%
\indextext{multibyte character|see{character, multibyte}}%
\indextext{object|seealso{object model}}%
\indextext{subobject|seealso{object model}}%
\indextext{derived class!most|see{most derived class}}%
\indextext{derived object!most|see{most derived object}}%
\indextext{program execution!as-if rule|see{as-if rule}}%
\indextext{observable behavior|see{behavior, observable}}%
\indextext{precedence of operator|see{operator, precedence of}}%
\indextext{order of evaluation in expression|see{expression, order of evaluation of}}%
\indextext{multiple threads|see{threads, multiple}}%

\rSec2[intro.compliance.general]{概述}

\pnum
\indextext{conformance requirements|(}%
\indextext{conformance requirements!general|(}%
\defn{可诊断规则} 集
包含本文档中的所有语法和语义规则,
但那些包含显式注释
“不需要诊断”或被描述为导致
“未定义行为”的规则除外。

\pnum
\indextext{conformance requirements!method of description}%
虽然本文档仅规定了对 \Cpp{}
实现的要求,但如果
将这些要求表述为对程序、程序部分或
程序执行的要求,通常更容易理解。此类要求具有以下含义:
\begin{itemize}
\item
如果程序不包含违反
\ref{lex} 到 \ref{\lastlibchapter} 以及 \ref{depr} 中规定的规则,
符合标准的实现应接受并正确执行
\begin{footnote}
“正确执行”可以包括未定义行为
和错误行为,具体取决于
正在处理的数据;参见 \ref{intro.defs} 和~\ref{intro.execution}。
\end{footnote}
该程序,
除非超出实现的限制(见下文)。
\item
\indextext{behavior!undefined}%
如果程序包含违反不需要诊断的规则,
则本文档对实现
在该程序方面的行为不做任何要求。

\item
\indextext{message!diagnostic}%
否则,如果程序包含
\begin{itemize}
\item
违反任何可诊断规则,
\item
具有 \tcode{\#warning} 预处理指令\iref{cpp.error} 的预处理翻译单元,或
\item
本文档中描述为“条件支持”的结构,
而实现不支持该结构,
\end{itemize}
符合标准的实现
应发出至少一条诊断消息。
\end{itemize}
\begin{note}
在模板实参推导和替换期间,
某些在其他上下文中需要诊断的结构
会被区别对待;
参见~\ref{temp.deduct}。
\end{note}
此外,符合标准的实现
不应接受
\begin{itemize}
\item
包含 \tcode{\#error} 预处理指令\iref{cpp.error} 的预处理翻译单元,或
\item
具有失败的 \grammarterm{static_assert-declaration}\iref{dcl.pre} 的翻译单元。
\end{itemize}
\pnum
\indextext{conformance requirements!library|(}%
\indextext{conformance requirements!classes}%
\indextext{conformance requirements!class templates}%
对于类和类模板,库条款指定了部分定义。私有成员\iref{class.access}
未指定,但每个实现都应提供它们以根据库条款中的描述完成定义。

\pnum
对于函数、函数模板、对象和值,库
条款指定了声明。实现应提供
与库条款中的描述一致的定义。

\pnum
\Cpp{} 翻译单元\iref{lex.phases}
通过包含适当的标准库头文件或导入
适当的标准库命名头单元\iref{using.headers} 来
获取对库中定义的名称的访问权限。

\pnum
库中的模板、类、函数和对象具有
外部链接\iref{basic.link}。在组合
翻译单元以形成完整的 \Cpp{} 程序\iref{lex.phases} 时,
实现会根据需要提供标准库实体的定义。%
\indextext{conformance requirements!library|)}

\pnum
定义了两种实现:\defnadj{宿主}{implementation} 和
\defnadj{独立}{implementation}。
独立
实现是一种可以在没有操作系统的情况下执行的实现。
宿主实现
支持本文档中描述的所有设施,而
独立实现
支持 \ref{lex} 到 \ref{\lastcorechapter} 中描述的整个 \Cpp{} 语言
以及 \ref{compliance} 中描述的库设施子集。

\pnum
如果可能且已知,鼓励实现记录
其在可以成功处理的程序的大小或复杂性方面的限制。
\ref{implimits} 列出了一些可能受到限制的数量以及
每个数量的潜在最小支持值。

\pnum
符合标准的实现可以具有扩展(包括
附加的库函数),前提是它们不改变任何良构程序
的行为。
要求实现诊断使用此类
根据本文档属于病构的扩展的程序。
但是,这样做之后,它们可以编译和执行此类程序。

\pnum
每个实现都应包含文档,该文档标识所有
它不支持的条件支持结构\indextext{behavior!conditionally-supported}
并定义所有区域特定的特征。
\begin{footnote}
本文档还定义了实现定义的行为;
参见~\ref{intro.abstract}。
\end{footnote}
\indextext{conformance requirements!general|)}%
\indextext{conformance requirements|)}%

\rSec2[intro.abstract]{抽象机器}

\pnum
\indextext{program execution|(}%
\indextext{program execution!abstract machine}%
本文档中的语义描述定义了一个
参数化的非确定性抽象机器。本文档
对符合标准的实现的结构不做任何要求。特别是,它们不需要复制或模拟
抽象机器的结构。
\indextext{as-if rule}%
\indextext{behavior!observable}%
相反,符合标准的实现需要模拟(仅)
如下所述的抽象机器的可观察行为。
\begin{footnote}
此规定
有时被称为“as-if”规则,因为实现可以自由
忽略本文档的任何要求,只要结果
\emph{如同} 遵守了该要求一样,就程序的可观察行为而言。例如,如果一个实际的
实现可以推断出表达式的某个部分的值
未使用,并且没有
\indextext{side effects}%
影响程序可观察行为的副作用
产生,则不需要对该部分求值。
\end{footnote}

\pnum
\indextext{behavior!implementation-defined}%
抽象机器的某些方面和操作在本文档中被描述为
实现定义的行为(例如,
\tcode{sizeof(int)})。这些构成了抽象机器的参数。
每个实现都应包含描述其在这些方面的特征
和行为的文档。
\begin{footnote}
本文档还包括
条件支持的结构和区域特定的行为。
参见~\ref{intro.compliance.general}。
\end{footnote}
此类文档应定义
与该实现相对应的抽象机器实例(下文称为
“相应的实例”)。

\pnum
\indextext{behavior!unspecified}%
抽象机器的某些其他方面和操作
在本文档中被描述为未指明行为(例如,
函数调用中参数的求值顺序\iref{expr.call})。
在可能的情况下,本文档
定义了一组允许的行为。这些
定义了抽象机器的非确定性方面。因此,一个
抽象机器的实例对于给定的程序和给定的输入
可以有多个可能的执行。

\pnum
\indextext{behavior!undefined}%
某些其他操作在本文档中被描述为
未定义行为(例如,尝试
修改 const 对象的效果)。
\begin{note}
本文档对
包含未定义行为的程序的行为不做任何要求。
\end{note}

\pnum
\indextext{program!well-formed}%
\indextext{behavior!observable}%
符合标准的实现执行良构程序应
产生与具有
相同程序和相同输入的抽象机器的相应实例的可能执行之一
相同的可观察行为。
\indextext{behavior!undefined}%
但是,如果任何此类执行包含未定义操作,则本文档对实现执行
具有该输入的该程序不做任何要求
(甚至不考虑第一个未定义
操作之前的操作)。
如果执行包含指定为具有错误行为的操作,
则允许实现在该操作之后的未指定时间发出诊断,
并且允许终止执行。

\pnum
\recommended
当执行此类操作时,实现应发出诊断。
\begin{note}
如果实现可以确定
在关于程序行为的一组特定于实现的假设下可到达错误行为,则它可以发出诊断消息,这可能会导致误报。

\end{note}

\pnum
\indextext{conformance requirements}%
对符合标准的实现的最低要求是:
\begin{itemize}
\item
通过 volatile 泛左值的访问严格按照
抽象机器的规则进行求值。
\item
在程序终止时,写入文件的所有数据应
与根据抽象语义执行程序
可能产生的结果之一相同。
\item
交互式设备的输入和输出动态
应以这样的方式进行:在程序等待输入之前,提示输出实际被传递。什么是交互式设备是
\impldef{交互式设备}。
\end{itemize}

这些统称为程序的
\defnx{可观察行为}{behavior!observable}。
\begin{note}
每个实现都可以定义抽象语义和实际
语义之间更严格的对应关系。
\end{note}
\indextext{program execution|)}%

\rSec1[intro.structure]{本文档的结构}

\pnum
\indextext{standard!structure of|(}%
\indextext{standard!structure of}%
\ref{lex} 到 \ref{\lastcorechapter} 描述了 \Cpp{} 编程
语言。该描述包括详细的语法规范,
其形式在 \ref{syntax} 中描述。为方便起见,\ref{gram}
重复了所有这些语法规范。

\pnum
\ref{\firstlibchapter} 到 \ref{\lastlibchapter} 和 \ref{depr}
(\defn{库条款})描述了 \Cpp{} 标准库。
该描述包括详细描述
构成库的实体和宏,
其形式在 \ref{library} 中描述。

\pnum
\ref{implimits} 建议了符合标准的
实现的容量下限。

\pnum
\ref{diff} 总结了 \Cpp{} 自首次
发布描述以来的演变,并详细解释了
\Cpp{} 和 C 之间的差异。 \Cpp{} 的某些特性仅出于兼容性
目的而存在;\ref{depr} 描述了这些特性。
\indextext{standard!structure of|)}

\rSec1[syntax]{语法表示}

\pnum
\indextext{notation!syntax|(}%
在本文档中使用的语法表示法中,语法
类别由 \fakegrammarterm{斜体, 无衬线} 字体表示,而文字单词
和字符以 \tcode{等宽} 字体表示。备选项
列在单独的行上,除非在少数情况下,由短语“one of”标记的一长串
备选项。如果备选项的文本太长而无法放在一行上,则文本将在后续行上继续,并从第一行缩进。
可选的终结符或非终结符由下标
“\opt{\relax}”表示,因此
\begin{ncbnf}
\terminal{\{} \opt{expression} \terminal{\}}
\end{ncbnf}
表示用大括号括起来的可选表达式。%

\pnum
语法类别的名称通常根据
以下规则选择:
\begin{itemize}
\item \fakegrammarterm{X-name} 是在确定其含义的上下文中使用标识符(例如,\grammarterm{class-name}、
\grammarterm{typedef-name})。
\item \fakegrammarterm{X-id} 是一个没有上下文相关含义的标识符
(例如,\grammarterm{qualified-id})。
\item \fakegrammarterm{X-seq} 是一个或多个没有中间分隔符的 \fakegrammarterm{X}(例如,\grammarterm{declaration-seq} 是一个声明序列)。
\item \fakegrammarterm{X-list} 是一个或多个由中间逗号分隔的 \fakegrammarterm{X}(例如,\grammarterm{identifier-list} 是一个由逗号分隔的标识符序列)。
\end{itemize}%
\indextext{notation!syntax|)}
